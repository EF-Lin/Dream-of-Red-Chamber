\chapter[饋土物顰卿念故里 訊家童鳳姐蓄陰謀]{饋土物顰卿念故里 訊家童鳳姐蓄陰謀\footnotemark\footnotemark}
\footnotetext{按:《程甲本》作「見土儀顰卿思故里 聞秘事鳳姐訊家童」}
\footnotetext{按:此回庚辰本缺。其他各本存在兩種類型文字,且出入較大。列、戚、甲辰本此回情節安排完整合理,但較為羅嗦拖沓,玩其文字,當非出於曹雪芹手筆,或系脂硯等人據曹雪芹殘稿補寫而成。楊本及程甲乙本一系文字比較簡練,顯系經過後人整理,且存在刪減過度及某些情節欠合理的問題。由於兩種版本文字多寡懸殊,無法互校,故本書第六十七回正文采用《列藏本》,另將《程甲本》文字附錄於後。}

\section*{列藏本}
話說尤三姐自戕之後,尤老娘以及尤二姐、賈珍、尤氏並賈蓉、賈璉等聞之,俱各不勝悲慟傷感,自不必說,忙着人治買棺木盛殮,送往城外埋葬。卻說柳湘蓮見尤三姐身亡,迷性不悟,尚有痴情眷戀,被道人數句偈言打破迷關,竟自削髮出家,跟隨瘋道人飄然而去,不知何往。後事暫且不表。

且說薛姨媽聞知湘蓮已說定了尤三姐為妻,心甚喜悅,正自高高興興要打算替他買房屋、治器用、辦妝奩,擇吉日迎娶過門等事,以報他救命之恩。忽有家中小廝見薛姨媽,告知尤三姐自戕與柳湘蓮出家的信息,心甚嘆息。正自猜疑是為什麼原故,時值寶釵從園子裡過來,薛姨媽便對寶釵說道:「我的兒,你聽見了沒有?你珍大嫂子的妹妹尤三姐,他不是已經許定了給你哥哥的義弟柳湘蓮的?這也很好。不知為什麼尤三姐自刎了,柳湘蓮也出了家了。真正奇怪的事,叫人意想不到!」寶釵聽了,並不在意,便說道:「俗語說的好,『天有不測風雲,人有旦夕禍福』。這也是他們前生命定,活該不是夫妻。媽所為的是因有救哥哥的一段好處,故諄諄感嘆。如果他兩人齊齊全全的,媽自然該替他料理,如今死的死了,出家的出了家了,依我說,也只好由他罷了。媽也不必為他們傷感,損了自己的身子。倒是自從哥哥起江南回來了一二十日,販了來的貨物,想來也該發完了,那同伴去的夥計們辛辛苦苦的,來回幾個月,媽同哥哥商議商議,也該請一請,酬謝酬謝才是。不然,倒叫他們看着無禮似的。」

母女正說之間,見薛蟠自外而入,眼中尚有淚痕未乾。一進門。便向他母親拍手說道:「媽,可知道柳大哥、尤三姐的事麼?」薛姨媽說:「我在園子裡聽見大家議論,正在這裡才和你妹子說這件公案呢。」薛蟠道:「這事可奇不奇?」薛姨媽說:「可是柳相公那樣一個年輕聰明的人,怎麼就一時糊塗跟着道士去了呢?我想他前世必是有夙緣的有根基的人,所以才容易聽得進這些度化他的話去。想你們相好了一場,他又無父母兄弟,隻身一人在此,你也該各處找一找才是。靠那跛足道士瘋瘋癲癲的,能往那裡遠去!左不過在這房前左右的廟裡寺里躲藏着罷咧。」薛蟠說:「何嘗不是呢。我一聽見這個信兒,就連忙帶了小廝們在各處尋找去,連個影兒也沒有。又去問人,人人都說不曾看見。我因如此,急的沒法,唯有望着西北上大哭了一場回來了。」說着,眼圈兒又紅上來了。薛姨媽說:「你既然找尋了沒有,把你作朋友的心也盡了。焉知他這一出家,不是得了好處去呢?你也不必太過慮了。一則張羅張羅買賣,二則把你自己娶媳婦應辦的事情,倒是早些料理料理。咱們家裡沒人手兒,竟是『笨雀兒先飛』,省得臨期丟三忘四的不齊全,令人笑話。再者,你妹妹才說,你也回家半個多月了,想貨物也該發完了,同你作買賣去的夥計們,也該設桌酒席請請他們,酬酬勞乏才是。他們固然是咱家約請的吃工食勞金的人,到底也算是外客,又陪着你走了一二千里的路程,受了四五個月的辛苦,而且在路上又替你擔了多少的驚怕沉重。」薛蟠聞聽,說:「媽說的很是,妹妹想得周到。我也這樣想來着,只因這些日子為各處發貨,鬧得頭暈。又為柳大哥的親事又忙了這幾日,反倒落了一個空,白張羅了一會子,倒把正經事都誤了。要不然,就定了明兒後兒下帖子請請罷。」薛姨媽道:「由你辦去罷。」話猶未了,外面小廝回說:「張管總的夥計着人送了兩個箱子來,說這是爺各自買的,不在貨賬裡面。本要早送來,因貨物箱子壓着,未得拿;昨日貨物發完了,所以今兒才送來了。」一面說,一面又見兩個小廝搬進了兩個夾板夾的大棕箱來。薛蟠一見,說:「噯喲,可是我怎麼就糊塗到這一步田地了!特特的給媽和妹妹帶來的東西都忘了,沒拿了家裡來,還是夥計送了來了。」寶釵說:「虧你才說還是特特的帶來的,還是這樣放了一二十日才送來,若不是特特的帶來,必定是要放到年底下才送進來呢。你也諸事太不留心了。」薛蟠笑道:「想是我在路上叫賊人把魂嚇掉了,還沒歸殼呢。」

說着,大家笑了一陣,便向回話的小廝說:「東西收下了,叫他們回去罷。」薛姨媽同寶釵忙問:「是什麼好東西,這樣捆着夾着的?」便命人挑了繩子,去了夾板,開了鎖看時,卻是些綢緞、綾錦、洋貨等家常應用之物。獨有寶釵他的那個箱子裡,除了筆、墨、硯、各色箋紙、香袋、香珠、扇子、扇墜、花粉、胭脂、頭油等物外,還有虎丘帶來的自行人、酒令兒、水銀灌的打筋斗的小小子,沙子燈,一出一出的泥人兒的戲,用青紗罩的匣子裝着,又有在虎丘上作的薛蟠的像,泥捏成的與薛蟠毫無相差,以及許多碎小玩意兒的東西。寶釵一見,滿心歡喜,便叫自己使的丫環來吩咐:「你將我的這個箱子與我拿了園子裡去,我好就近從那邊送送人。」說着,便起身來,告辭母親,往園子裡來了。這裡薛姨媽將自己這個箱子裡的東西取出,一分一分的打點清楚,着同喜丫頭送往賈母並王夫人等處去不講。

且說寶釵隨着箱子到了自己房中,將東西逐件逐件的過了目,除將自己留用之外,遂一分一分配合妥當:也有送筆、墨、紙、硯的,也有送香袋、扇子、香墜的,也有送脂粉、頭油的,有單送玩意兒的;酌量其人分辦。只有黛玉的比別人不同,比眾人加厚一倍。一一打點完畢,使鶯兒同一個老婆子跟着,送往各處。

其李紈、寶玉等以及諸人,不過收了東西,賞賜來使,皆說些見面再謝等語而已。惟有林黛玉他見江南家鄉之物,反自觸物傷情,因想起他的父母來了。便對着這些東西,揮淚自嘆,暗想:「我乃江南之人,父母雙亡,又無兄弟,隻身一人,可憐寄居外祖母家中,而且又多疾病,除外祖母以及舅母、姐妹看問外,那裡還有一個姓林的親人來看望看望,給我帶些土物來。使我送送人,粧粧臉面也好。可見人若無至親骨肉手足,是最寂寞、極冷清、極寒苦,沒趣味的!」想到這裡,不覺就大傷起心來了。紫鵑他乃伏侍黛玉多年,朝夕不離左右的,深知黛玉的心腹:他為見了江南故土之物,因感動了心懷,追思親人的原故。但不敢說破,只在一旁勸說道:「姑娘的身子多病,早晚尚服丸藥,這兩日看着不過比那些日子略飲食好些,精神壯一點兒,還算不得十分大好。今兒寶姑娘送來這些東西,可見寶姑娘素日看姑娘甚重,姑娘看着該歡喜才是,為什麼反倒傷感。這不是寶姑娘送東西為的是叫姑娘歡喜,這反倒是招姑娘煩惱了不成?若令寶姑娘知道了,怎麼臉上下得來呢?再姑娘也要細想一想,老太太、太太們為姑娘的病症千方百計請好大夫診脈配藥調治,所為的是姑娘的病急好。這如今才好些,又這樣哭哭啼啼的,豈不是自己糟蹋自己的身子,不肯叫老太太看着歡喜?難道說姑娘這個病,不是因素日從憂慮過度上傷多了氣血得的麼?姑娘的千金貴體別自己看輕了。」紫鵑正在這裡勸解黛玉,只聽見小丫頭子在院內說:「寶二爺來了。」紫鵑忙說:「快請。」

話猶未畢,只見寶玉已進房來了。黛玉讓坐畢,寶玉見黛玉淚痕滿面,便問:「妹妹,又是誰得罪了你了?你兩眼都哭得紅了,是為什麼?」黛玉不回答。旁邊紫鵑將嘴向床里一扭,寶玉會意,便往床里一看,見堆着許多東西,就知是寶釵送來的,便笑着取笑說道:「好東西,想是妹妹要開雜貨鋪麼?擺着這些東西作什麼?」黛玉只是不理。紫鵑說:「二爺還提東西呢。因寶姑娘送了些東西來,我們姑娘一看,就傷心哭起來了。我正在這裡好勸歹勸,總勸不住呢。而且又是才吃了飯,若只管哭,大發了,再吐了,犯了舊病,可不叫老太太罵死了我們麼?倒是二爺來的很好,替我們勸一勸。」寶玉他本是聰明人,而且一心總留意在黛玉身上最重,所以深知黛玉之為人心細心窄,而又多心要強,不落人後,因見了人家哥哥自江南帶了東西來送人,又系故鄉之物,勾想起別的痛腸來,是以傷感是實。這是寶玉他心裡揣摩黛玉的心病,卻不肯明明說出,恐黛玉越發動情,乃笑道:「你們姑娘的原故不為別的,為的是寶姑娘送來的東西少,所以生氣傷心。妹妹,你放心!等我明年往江南去,與你多多的帶兩船來,省得你淌眼抹淚的。」黛玉聽了這話,不由「嗤」的一聲笑了,忙說道:「我憑他怎麼沒見過世面,也到不了這一步田地上,因送的東西少,就生氣傷心。我也不是兩三歲的小孩子,你也忒把人看得平常小氣了。我有我的原故,你那裡知道。」說着說着,眼淚又流下來了。寶玉忙移至床上,挨黛玉坐下,將那些東西一件一件的拿起來,擺弄着細瞧,故意問:「這是什麼,叫什麼名字?那是怎麼做的,這樣齊整?這是什麼,要他做什麼使用?妹妹,你瞧,這一件可以擺在書閣兒上作陳設,那件放在條案上當古董兒倒好呢!」一味的將這些沒要緊的話來支吾搭訕了一會,黛玉見寶玉那些呆樣子,問東問西的,招人可笑,稍將煩惱丟開,略有些喜笑之意。寶玉見他有些喜色,便說道:「寶姐姐送東西來給咱們,我想着,咱們也該到他那裡道個謝去才是,不知妹妹可去不去?」黛玉原不願意為送些東西來就特特的道謝去,不過一時見了,說一聲就完了。今被寶玉說得有理難以推託,無奈只得同寶玉去了。這且不提。

且說薛蟠聽了母親之言,急忙下請帖,置辦酒筵。張羅了一日,果於次日,三四位夥計,俱各到齊。未免說了些店內發貨、帳目之事畢,列席讓坐,薛蟠與各位奉酒酬勞。裡面薛姨媽又着人出來致謝道乏,畢,內有一位問道:「今日席上怎麼少柳大哥不出來?想是東家忘了,沒請麼?」薛蟠聞聽,把眉一皺,嘆了一口氣,說道:「休提,休提,想來眾位不知深情。若說起此人,真真可嘆!於一二日前,忽被一個瘋道士度化的出了家,跟着他去了。你們眾位聽一聽,可奇不奇?」眾人說道:「我們在店內也聽見外面人吵嚷,說有一個道士三言兩語把一個俗家子弟度了去了,又聞說一陣風颳了去了,又說駕着一片雲彩去了,紛紛議論不一。我們也因發貨事忙,那裡有工夫當正經事,也沒去細問細打聽,到如今還是似信不信的。今聽此言,那道士度化的原來就是柳大哥麼?早知是他,我們大家也該勸解勸解。憑他怎麼,也不容他去。噯,又少了一個有趣兒的好朋友了!實實在在的可惜可嘆。也怨不得東家你心裡不爽快。想他那樣一個伶俐人,未必是真跟了道士去罷。柳大哥他會些武藝,又有力量,或者看破了道士有些什麼妖術邪法的破綻出來,故意假跟了他去,在背地裡擺布他也未可知。」薛蟠說:「誰知道,果能如此,倒好罷咧,世上也少一個妖言惑眾的人了。」眾人道:「難道你知道了的時候,也沒尋找他去不成?」薛蟠說:「城裡城外,那裡沒有找到!不怕你們笑話,我還哭了一場呢。」言畢,只是長吁短嘆,無精打彩的,不像往日高興頑笑,讓酒暢飲。席上雖設了些雞鵝魚鴨,山珍海味,美品佳餚,怎奈東家皺眉嘆氣,眾夥計看此光景,不便久坐,不過隨便喝了幾鍾酒,吃了些飯食,就都散了。這也不提。

且說寶玉拉了黛玉至寶釵處來道謝。彼此見面,未免說幾句客言套語。黛玉便對寶釵說道:「大哥哥辛辛苦苦的能帶了多少東西來,擱得住送我們這些處,你還剩什麼呢?」寶玉說:「可是這話呢。」寶釵笑道:「東西不是什麼好的,不過是遠路帶來的土物兒,大家看着略覺新鮮似的。我剩不剩什麼要緊,我如今果愛什麼,今年雖然不剩,明年我哥哥去時,再叫他給我帶些個來,有什麼難呢?」寶玉聽說,忙笑道:「明年再帶了什麼來,我們還要姐姐送我們呢。可別忘了我們!」黛玉說:「你要,你只管說你要,不必拉扯上『我們』不『我們』的字眼,姐姐瞧寶哥哥不是給姐姐來道謝,竟是又要定下明年的東西來了。」寶玉笑說:「我要出來,難道沒有你一分兒不成?你不知道幫着說,反倒說起這散話來了。」大家聽了,笑了一陣。寶釵問:「你二人如何來得這樣巧,是誰會誰去的?」寶玉說:「休提,我因姐姐送我東西,想來林妹妹也必有,我想要來道謝,想林妹妹也必來道謝,故此我就到他房裡會了他一同要到這裡來。誰知到了他家,他正在屋裡傷心落淚,也不知是為什麼這樣愛哭。」寶玉剛說到「落淚」兩字,見黛玉瞪了他一眼,恐他往下還說。寶玉會意,隨即便換過口來說道:「林妹妹這幾日因身上不爽快,恐怕又病扳嘴,故此着急落淚。我勸解了一會子,才來了。一則道謝;二則省的叫他一個人在房裡坐着只是發悶。」寶釵說:「妹妹怕病悶,固然是正理,也不過是在那飲食起居、穿脫衣服冷熱上加些小心就是了,為什麼傷起心來呢?妹妹,你難道不知傷心難免不傷氣血精神,把要緊的傷了,反倒要受病的罷咧。妹妹你細想想。」黛玉說:「姐姐說的很是。我何嘗自己不知道呢,只因我這幾年,姐姐是看見的,那一年不病一兩場?病的我怕怕的了。見了藥,吃了見效不見效,一聞見,先就頭疼發噁心,怎麼不叫我怕病呢?」寶釵說:「雖然如此說,卻也不該傷心,倒是覺着身上不爽快,反自己勉強扎掙着出來,各處走走逛逛,把心鬆散鬆散,比在屋裡悶坐着還強呢。傷心是自己添病的大毛病。我那兩日不時覺着發懶,渾身乏倦,只是要歪着,心裡也是為時氣不好,怕病,因此偏扭着他,尋些事情作作,一般里也混過去了。妹妹別惱我說,越怕越有鬼。」寶玉聽說,忙問道:「寶姐姐,鬼在那裡呢?我怎麼看不見一個兒?」惹得眾人哄聲大笑。寶釵道:「呆小爺,這是比喻的話,那裡真有鬼呢!認真的果有鬼,你又該駭哭了。」黛玉因此笑道:「姐姐說的很是。很該說他,誰叫他嘴快!」寶玉說:「有人說我的不是,你就樂了。你這會子心裡也不懊惱了,咱們也該走罷。」於是彼此又說笑了一回,二人辭了寶釵出來。寶玉仍把黛玉送至瀟湘館門首,自己回家。這且不提。

且說趙姨娘因見寶釵送環哥之物,忙忙接下,心中甚喜,滿嘴誇獎:「人人都說寶姑娘會行事,很大方,今日看來,果然不錯。他哥哥能帶了多少東西來,他挨家送到,並不遺漏一處,也不露出誰薄誰厚,連我們搭拉嘴子,他都想到,實在的可敬。若是林姑娘——也罷麼,也沒人給他送東西帶什麼來;即或有人帶了來,他也只是揀着那有勢力、有體面的人頭兒跟前才送去,那裡還臨的到我們娘兒們身上呢!可見人會行事,真真的露着各別另樣的好。」趙姨娘因環哥兒得了東西,深為得意,不住的托在掌上擺弄瞧看一會。想寶釵乃係王夫人之表侄女,特要在王夫人跟前賣好兒。自己疊疊歇歇的拿着那東西,走至王夫人房中,站在一旁說道:「這是他寶姑娘才給環哥他兄弟送來的。他年輕輕的人想的周到,我還給了送東西的小ㄚ頭二百錢。聽見說姨太太也給太太送來了,不知是什麼東西?你們瞧瞧這一個門裡頭就是兩分兒,能有多少呢?怪不的老太太同太太都誇他疼他,果然招人愛。」說着,將抱的東西遞過去與王夫人瞧,誰知王夫人頭也沒抬,手也沒伸,只口內說了一聲「好,給環哥兒玩罷咧」,並無正眼看一看。趙姨娘因招了一鼻子灰,滿肚氣惱,無精打彩的回至自己房中,將東西丟在一邊,說了許多的勞兒三、巴兒四,不着要的一套閒話;也無人問他,他卻自己咕嘟着嘴,一邊子坐着。可見趙姨娘為人小器糊塗,饒得了東西,反說許多令人不入耳生厭的閒話,也怨不得探春生氣,看不起他。閒話休提。

且說寶釵送東西的ㄚ頭回來,說:「也有道謝的,也有賞賜的,獨有給巧姐兒的那一分兒,仍舊拿回來了。」寶釵一見,不知何意,便問:「為什麼這一分兒沒送去呢,還是送了去沒收呢?」鶯兒說:「我方才給環哥兒送東西的時候,見璉二奶奶往老太太房裡去了。我想,璉二奶奶不在家,知道交給誰呢,所以沒有送去。」寶釵說:「你也太糊塗了。二奶奶不在家,難道平兒、豐兒也不在家不成?你只管交給他們收下,等二奶奶回來,自有他們告訴就是了,必定要你當面交給才算麼?」鶯兒聽了,復又拿着東西出了園子,往鳳姐處去。在路上走着,便對拿東西的老婆子說:「早知道一就事兒送了去不完了,省得又跑這一趟。」老婆子說:「閒着也是白閒着,藉此出來逛逛也好罷咧。只是姑娘你今日來回各處走了好些路兒,想是不慣,乏了,咱們送了這個,可就完了,一打總兒再歇着。」兩人說着話,到了鳳姐處,送了東西,回來見寶釵。寶釵問道:「你見了璉二奶奶沒有?」鶯兒說:「我沒有見。」寶釵說:「想是二奶奶還沒回來麼?」ㄚ頭說:「回是回來了。因豐兒對我說:『二奶奶自老太太屋裡回房來,不似往日歡天喜地的,一臉的怒氣,叫了平兒去,唧唧咕咕的說話,也不叫人聽見。連我都攆出來了,你不必去見,等我替你回一聲兒就是了。』因此便着豐兒他拿進去,回了出來說:『二奶奶說,給你們姑娘道生受。』賞了我們一吊錢,我就回來了。」寶釵聽了,自己納了一會子悶,也想不出鳳姐是為什麼有氣。這也不表。

且說襲人見寶玉回來,便問:「你怎麼不逛就回來了?你原說約着林姑娘,你們兩個同到寶姑娘處道謝去,可去了沒有?」寶玉說:「你別問,我原說是要會林姑娘同去的,誰知到了他家,他在房裡守着東西很很的不自在呢。我也知道林姑娘的那些原原故故的,又不好直問他,又不好說他,只裝不知道兒,搭訕着說別的寬解了他一會子,才好了。然後方拉了他同到了寶姐姐那裡道了謝,說了一會子閒話,方散了。我又送他到家,我才回來了。」襲人說:「你看送林姑娘的東西,比送你的是多是少,還是一樣呢?」寶玉說:「比送我的多着一兩倍呢。」襲人說:「這才是明白人,會行事。寶姑娘他想別的姊妹等都有親的熱的跟着,有人送東西,唯有林姑娘離家二三千里地遠,又無有一個親人在這裡,那有人送東西。況且他們兩個不但是親戚,還是乾姐妹,難道你不知道林姑娘去年曾認過薛姨太太作乾媽的?論理多給他些也是該的。」

寶玉笑說:「你就是會評事的一個公道老兒。」說着話兒,便叫小丫頭取了拐枕來,要在床上歪着。襲人說:「你不出去了?我有一句話告訴你。」寶玉便問:「什麼話?」襲人說:「素日璉二奶奶待我很好,你是知道的。他自從病了一大場之後,如今又好了。我早就想着要到那裡看看去,只因為璉二爺在家不方便,始終總沒有去,聞說璉二爺不在家,你今日又不往那裡去,而且初秋天氣,不冷不熱,一則看二奶奶,盡個禮,省得日後見了受他的數落;二則藉此也逛一逛。你同他們看着家,我去去就來。」晴雯說:「這卻是該的,難得這個巧空兒。」寶玉說:「我才為他議論寶姑娘,誇他是個公道人,這一件事行的,又是一個周到人了。」襲人笑道:「好小爺,你也不用誇我,你只在家同他們好生玩;好歹別睡覺,看睡出病來,又是我擔沉重。」寶玉說:「我知道了,你只管去罷。」言畢,襲人遂到自己房裡,換了兩件新鮮衣服,拿着把兒鏡照着,抿了抿頭,勻了勻臉上脂粉,步出下房。復又囑咐了晴雯、麝月幾句話,便出了怡紅院。

來至沁芳橋上立住,往四下里觀看那園中景致。時值秋令,秋蟬鳴於樹,草蟲鳴於野;見這石榴花也開敗了,荷葉也將殘上來了,倒是芙蓉近着河邊,都發了紅鋪鋪的咕嘟子,襯着碧綠的葉兒,倒令人可愛。一壁里瞧着,一壁里下了橋。走了不遠,迎見李紈房裡使喚的丫頭素雲,跟着個老婆子,手裡捧着一個洋漆盒兒走來。襲人便問:「往那裡去?送的是什麼東西?」素雲說:「這是我們奶奶給三姑娘送去的菱角、雞頭。」襲人說:「這個東西,還是咱們園子裡河內采的,還是外頭買來的呢?」素雲說:「這是我們房裡使喚的劉媽媽,他告假瞧親戚去帶來的,孝敬奶奶。因三姑娘在我們那裡坐着看見了,我們奶奶叫人剝了讓他吃。他說:『才喝了熱茶了,不吃,一會子再吃罷。』故此給三姑娘送了家去。」言畢,各自分路走了。

襲人遠遠的看見那邊葡萄架底下,有一個人拿着撣子在那裡動手動腳的,因迎着日光,看不真切。至離得不遠,那祝老婆子見了襲人,便笑嘻嘻的迎上來,說道:「姑娘今日怎麼得工夫出來閒逛,往那裡去?」襲人說:「我那裡還得工夫來逛,我往璉二奶奶家瞧瞧去。你在這裡做什麼呢?」那祝婆子說:「我在這裡趕馬蜂呢。今年三伏里的雨水少,不知怎麼,這些果木樹上長蟲子,把果子吃得巴拉眼睛的,掉了好些下來,可惜了兒的白扔了!就是這葡萄,剛成了珠兒,怪好看的,那馬蜂、蜜蜂兒滿滿的圍着來蚛,都咬破了。這還罷了,喜鵲、雀兒,他也來吃這個葡萄。還有這樣一個毛病兒,無論雀兒蟲兒,一嘟嚕上只咬破三五個,那破的水淌到好的上頭,連這一嘟嚕都是要爛的。這些雀兒、馬蜂可惡着呢,故此我在這裡趕。姑娘你瞧,咱們說話的空兒沒趕,就蚛了許多上來了。」襲人道:「你就是不住手的趕,也趕不了許多;你剛趕了這裡,那裡又來了。倒是告訴買辦說,叫他多多的作些冷布口袋來,一嘟嚕一嘟嚕的套上,免得翎禽草蟲糟蹋,而且又透風,捂不壞。」婆子笑道:「倒是姑娘說的是。我今年才管上,那裡就知道這些巧法兒呢。」

襲人說:「如今這園子裡這些果品有好些種,到是那樣先熟的快些?」老祝婆子說:「如今才入七月的門,果子都是才紅上來,要是好吃,想來還得月盡頭兒才熟透了呢。姑娘不信,我摘一個給姑娘嘗嘗。」襲人正色說道:「這那裡使得?不但沒熟吃不得,就是熟了,一則沒有供鮮,二則主子們尚然沒吃,咱們如何先吃得呢?你是這府里的陳人,難道連這個規矩也不曉得麼?」老婆子忙笑道:「姑娘說得有理。我因為姑娘問我,我白這樣說。」心內暗說道:「夠了!我方才幸虧是在這裡趕馬蜂,若是順着手兒摘一個嘗嘗,叫他看見,還了得了!」襲人說:「我方才告訴你要口袋的話,你就回一回二奶奶,叫管事的作去罷。」言畢,遂一直的出了園子的門,就到鳳姐這裡來了。

正是鳳姐與平兒議論賈璉之事。因見襲人他是輕易不來之人,又不知是有什麼事情,便連忙止住話語,勉強帶笑說道:「貴人從那陣風兒刮了我們這個賤地來了?」襲人笑說:「我就知道奶奶見了我,是必定要先麻煩我一頓的,我有什麼說的呢!但是奶奶欠安,本心惦着要過來請請安,頭一件,璉二爺在家不便,二則奶奶在病中,又怕嫌煩,故未敢來。想奶奶素日疼愛我的那個分兒上,自必是體諒我,再不肯惱我的。」鳳姐笑道:「寶兄弟屋裡雖然人多,也就靠着你一個兒照看,也實在的離不開。我常聽見平兒告訴我,說你背地裡還惦着我,常問,我聽見就喜歡得的什麼似的。今日見了你,我還要給你道謝呢,我還捨得麻煩你嗎?我的姑娘!」襲人說:「我的奶奶,若是這樣說,這就是真疼我了。」鳳姐拉了襲人的手,讓他坐下。襲人那裡肯坐,讓之再三,方在挨炕沿腳踏上坐了。

平兒忙自己端了茶來。襲人說:「你叫小人兒們端罷,勞動姑娘我倒不安。」一面站起,接過茶來吃着,一面回頭看見床沿上放着一個活計簸羅兒,內裝着一個大紅洋錦的小兜肚,襲人說:「奶奶一天七事八事的,忙的不了,還有工夫作活計麼?」鳳姐說:「我本來就不會作什麼,如今病了才好,又兼着家務事鬧個不清,那裡還有工夫做這些呢?要緊要緊的我都丟開了。這是我往老太太屋裡請安去,正遇見薛姨太太送老太太這個錦,老太太說:『這個花紅柳綠的,倒對給小孩子們做小衣小裳兒的,穿着倒好頑呢!』因此我就問老祖宗討了來了。還惹的老祖宗說了好些頑話,說我是老太太的命中小人,見了什麼要什麼,見了什麼拿什麼。惹得眾人都笑了。你是知道我是臉皮兒厚、不怕說的人,老祖宗只管說,我只管裝聽不見,拿着就走。所以才交給平兒,先給巧姐兒做件小兜肚穿着頑,剩下的等消閒有工夫再作別的。」

襲人聽畢,笑道:「也就是奶奶,才能夠慪的老祖宗喜歡罷咧。」伸手拿起來一看,便夸道:「果然好看!各樣顏色都有。好材料也須得這樣巧手的人做才對。況又是巧姐兒他穿的,抱了出去,誰不多看一看。」又問道:「巧姐兒那裡去了?我怎麼這半日沒見他?」平兒說:「方才寶姑娘那裡送了些頑的東西來,他一見了很希罕,就擺弄着頑了好一會子,他奶媽兒才抱了出去,想是乏了,睡覺去了。」襲人說:「巧姐兒比先前自然越發會頑了。」平兒說:「小臉蛋子吃得銀盆似的,見了人就趕着笑,再不得罪人,真真是我奶奶的解悶的寶貝疙瘩兒。」鳳姐便問:「寶兄弟在家作什麼呢?」襲人笑道:「我只求他同晴雯他們看家,我才告了假來了。可是呢!只顧說話,我也來了好大半天了,要回去了。別叫寶玉在家裡抱怨,說我屁股沉,到那裡就坐住了。」說着,便立起身來告辭,回怡紅院來了。這也不提。

且說鳳姐見平兒送出襲人回來,復又把平兒叫入房中,追問前事,越說越氣,說道:「二爺在外邊偷娶老婆,你說你是聽見二門上的小廝們說的。到底是那一個說的呢?」平兒說:「是旺兒他說的。」鳳姐便命人把旺兒叫來,問道:「你二爺在外邊買房子娶小老婆,你知道麼?」旺兒說:「小的終日在二門上聽差,如何知道二爺的事,這是聽見興兒告訴的。」鳳姐說:「興兒是幾時告訴你的?」旺兒說:「還是二爺沒起身的頭裡告訴我的。」鳳姐又問:「興兒在那裡呢?」旺兒說:「興兒在新二奶奶那裡呢。」鳳姐聞聽,滿腔怒氣,啐了一口,罵道:「下作猴兒崽子!什麼是『新奶奶』、『舊奶奶』,你就私自封了奶奶了?滿嘴裡胡說,這就該打嘴巴。」又問:「興兒他是跟二爺的人,他怎麼沒有跟了二爺去呢?」旺兒說:「特留下他在家裡照看尤二姐,故此未曾跟了去。」鳳姐聽說,忙得一疊連聲命旺兒:「快把興兒叫了來!」

旺兒忙忙的跑了出去,見了興兒只說:「二奶奶叫你呢。」興兒正在外邊同小人兒們頑笑,聽見叫他,妙在也不問旺兒「二奶奶叫我做什麼」,便跟了旺兒,急急忙忙的來至二門前。回明進去,見了鳳姐,請了安,旁邊侍立。鳳姐一見,便先瞪了兩眼,問道:「你們主子奴才在外面幹的好事!你們打量我是呆瓜,不知道?你是緊跟二爺的人,自必深知根由。你須細細的對我實說,稍有一些兒隱瞞撒謊,我將你的腿打折了!」興兒忙跪下磕頭,說:「奶奶問的是什麼事,是我同爺干的?」鳳姐罵道:「好小雜種!你還敢來支吾我?我問你,二爺在外邊,怎麼就說成了尤二姐?怎麼買房子、治傢伙?怎麼娶了過來?一五一十的說個明白,饒你的狗命!」

興兒聽說,仔細想了一想:「此事二府皆知,就是瞞着老爺、太太、老太太同二奶奶不知道,終久也是要知道的。我如今何苦來瞞着,不如告訴了他,省得挨眼前打,受委屈。」再興兒一則年幼,不知事的輕重;二則素日又知道鳳姐是個烈口子,連二爺還懼怕他五分;三則此事原是二爺同珍大爺、蓉哥他叔侄弟兄商量着辦的,與自己無干。故此把主意想定,壯着膽子,跪下說道:「奶奶別生氣,等奴才回稟奶奶聽:只因那府里的大老爺的喪事上穿孝,不知二爺怎麼看見過尤二姐幾次,大約就看中了,動了要說的心。故此先同蓉哥商議,求蓉哥替二爺從中調停辦理,作了媒人說合,事成之後,還許下謝候的禮。蓉哥滿應,將此話轉告訴了珍大爺;珍大爺告訴了珍大奶奶和尤老娘。尤老娘很願意,但說是:『二姐從小兒已許過張家為媳,如何又許二爺呢?恐張家知道,生出事來不妥當。』珍大爺笑道:『這算什麼大事,交給我!便說那張姓的小子,本是個窮苦破落戶,那裡見得多給他幾兩銀子,叫他寫張退親的休書,就完了。』後來,果然找了姓張的來,如此說明,寫了休書,給了銀子去了。二爺聞知,才放心大膽的說定了。又恐怕奶奶知道,攔擋不依,所以在外邊咱們後身兒買了幾間房子,治了東西,就娶過來了。珍大爺還給了兩口人使喚。二爺時常推說給老爺辦事,又說給珍大爺張羅事,都是些支吾的謊話,竟是在外頭住着。從前原是娘兒三個住着,還要商量給尤三姐說人家,又許下厚聘嫁他;如今尤三姐也死了,只剩下尤老娘跟着尤二姐住着作伴兒呢。這是一往從前的實話,並不敢隱瞞一句。」說畢,復又磕頭。

鳳姐聽了這一篇言詞,只氣得痴呆了半天,面如金紙,兩隻吊稍子眼越發直豎起來了,渾身亂戰。半晌,連話也說不上來,只是發怔。猛一低頭,見興兒在地下跪着,便說道:「這也沒你的大不是,但只是二爺在外邊行這樣的事,你也該早些告訴我才是。這卻很該打,因你肯實說,不撒謊,且饒恕你這一次。」興兒說:「未能早回奶奶,這是奴才該死!」便叩頭有聲。鳳姐說:「你去罷。」興兒才立起身要走,鳳姐又說:「叫你時,須要快來,不可遠去。」興兒連連答應了幾個「是」,就出去了。到外面伸了伸舌頭,說:「夠了我的了,差一差兒沒有捱一頓好打。」暗自後悔不該告訴旺兒,又愁二爺回來怎麼見,各自害怕。這也不提。

且說鳳姐見興兒出去,回頭向平兒說:「方才興兒說的話,你都聽見了沒有?」平兒說:「我都聽見了。」鳳姐說:「天下那有這樣沒臉的男人!吃着碗裡,看着鍋里,見一個,愛一個,真成了餵不飽的狗,實在的是個棄舊迎新的壞貨。只是可惜這五六品的頂戴給他!他別想着俗語說的『家花那有野花香』的話,他要信了這個話,可就大錯了。多早晚在外面鬧一個很沒臉、親戚朋友見不得的事出來,他才罷手呢!」平兒一旁勸道:「奶奶生氣,卻是該的。但奶奶身子才好了,也不可過於氣惱。看二爺自從鮑二的女人那一件事之後,倒很收了心,好了呢,如今為什麼又干起這樣事來?這都是珍大爺他的不是。」鳳姐說:「珍大爺固然有不是,也總因咱們那位下作不堪的爺他眼饞,人家才引誘他罷咧。俗語說的『牛不吃水,也強按頭麼?』」平兒說:「珍大爺幹這樣事,珍大奶奶也該攔着不依才是。」鳳姐說:「可是這話咧!珍大奶奶也不想一想,把一個妹子要許幾家子弟才好,先許了姓張的,今又嫁了姓賈的;天下的男人都死絕了,都嫁了賈家來!難道賈家的衣飯這樣好不成?這不是說幸而那一個沒臉的尤三姐知道好歹,早早兒的死了,若是不死,將來不是嫁寶玉,就是嫁環哥兒呢。總也不給那妹子留一些兒體面,叫妹子日後怎麼抬頭豎臉的見人呢?妹子好歹也罷咧!那妹子本來也不是他親的,而且聽見說原是個混帳爛桃。難道珍大奶奶現做着命婦,家中有這樣一個打嘴現世的妹子,也不知道羞臊,躲避着些,反到大面兒上揚名打鼓的,在這門裡丟醜,也不怕人笑話麼?再者,珍大爺也是作官的人,別的律例不知道也罷了,連個服中娶親、停妻再娶使不得的規矩,他也不知道不成?你替他細想一想,他幹的這件事,是疼兄弟,還是害兄弟呢?」平兒說:「珍大爺只顧眼前,叫兄弟喜歡,也不管日後的輕重干係了。」鳳姐兒冷笑道:「這是什麼『叫兄弟喜歡』,這是給他毒藥吃呢!若論親叔伯弟兄中,他年紀又最大,又居長,不知教導兄弟學好,反引誘兄弟學不長進,擔罪名兒,日後鬧出事來,他在一邊缸沿兒上站着看熱鬧,真真我要罵也罵不出口來。再者,他那邊府里的醜事壞名兒,已經叫人聽不上了,必定也叫兄弟學他一樣,才好顯不出他的丑來。這是什麼作哥哥的道理?倒不如撒泡尿浸死了,替大老爺死了倒罷咧,活着作什麼呢!你瞧東府里大老爺那樣厚德,吃齋念佛行善,怎麼反得了這樣一個兒子孫子?大概是好風水都叫他老人家一個人全拔盡了。」平兒說:「想來不錯。若不然,怎麼這樣差着格兒呢?」鳳姐說:「這件事幸而老太太、老爺、太太不知道,倘或吹到這幾位耳朵里去,不但咱們那沒出息的二爺捱打受罵,就是珍大爺和珍大奶奶也保不住要吃不了要兜着走呢!」連說帶詈,直鬧了半天,連午飯也推頭疼,沒過去吃。

平兒看此光景越說越氣,勸道:「奶奶也煞一煞氣,事從緩來,等二爺回來,慢慢的再商量就是了。」鳳姐聽了此言,便從鼻孔內哼了兩聲,冷笑道:「好罷咧,等爺回來,可就遲了!」平兒便跪在地下,再三苦勸,安慰了一會子,鳳姐才略消了些氣惱。喝了口茶,喘息了良久,便要了拐枕,歪在床上,閉着眼睛打主意。平兒見鳳姐兒躺着,方退出去。偏有不懂眼的幾起子回事的人來,都被豐兒攆出去了。又有賈母處着瑪瑙來問:「二奶奶為什麼不吃飯?老太太不放心,着我來瞧來了。」鳳姐知是賈母處打發人來,遂勉強起來,說:「我白有些頭疼,並沒別的病,請老太太放心。我已經躺了一躺兒,好了。」言畢,打發來人去後,卻自己一個人將前事從頭至尾細細的盤算多時,得了一個「一計害三賢」的狠主意出來。自己暗想:須得如此如此方妥。主意已定,也不告訴平兒,反外面作出嘻笑自若、無事的光景,並不露出惱恨妒嫉之意。於是叫丫頭傳了來旺來吩咐,令他明日傳喚匠役人等,收拾東廂房,裱糊鋪設等語。平兒與眾人皆不知為何緣故。要知端的,且看下回分解。

\section*{程甲本}
話說尤三姐自戕之後,尤老娘以及尤二姐、賈珍、尤氏並賈蓉、賈璉等俱不勝悲慟傷感,忙着買棺盛殮,送往城外埋葬。柳湘蓮見尤三姐身亡,迷性不悟,尚有癡情眷戀,卻被道人數句偈言打破迷關,竟自削髮出家,隨一瘋道人飄然而去,不知何往。

薛姨媽聞知湘蓮已說定了尤三姐,正打算替他買房置器,擇日迎娶過門,以報他救命之恩。忽有家中小廝吿知尤三姐自戕與柳湘蓮出家之事,心甚嘆息。時值寶釵從園中過來,聽了這些話,並不在意,乃勸道:「俗語說的好,『天有不測風雲,人有旦夕禍福』。這也是他們前生命定。前日媽媽為他救了哥哥,商量著替他料理,如今已經死的死了,走的走了,依我說,也只好由他罷了。媽媽也不必為他們傷感了。倒是自從哥哥打江南回來了一二十日,販了來的貨物,想來也該發完了,媽媽和哥哥商議商議,酬謝酬謝那同去的那張德輝才是。伙計們辛辛苦苦的,回來幾個月了,也該請一請,別叫人家看著無禮似的。」

母女正說話間,見薛蟠自外而入,眼中尚有淚痕,一進門來,便向他母親拍手說道:「媽媽可知道柳二哥尤三姐的事麼?」薛姨媽說:「我才聽見說,正在這裡合你妹妹說這件公案呢。」薛蟠道:「媽媽可聽見說湘蓮跟著一個道士出了家了麼?」薛姨媽道:「這越發奇了。怎麼柳相公那樣一個年輕的聰明人,一時胡塗了,就跟著道士去了呢?我想你們好了一場,他又無父母兄弟,隻身一人在此,你該各處找找他才是。靠那道士,能往那裡遠去?左不過是在這方近左右的廟裡寺裡罷了。」薛蟠說:「何嘗不是呢?我一聽見這個信兒,就連忙帶了小廝們在各處尋找,連一個影兒也沒有。又去問人,都說沒看見。」

薛姨媽說:「你既找尋過,沒有,也算把你做朋友的心盡了。再者,你妹妹才說你也回家半個多月了,想貨物也該發完了,也該擺桌酒,給張德輝和夥計們,道道乏才是。」薛蟠聽說,便道:「媽媽說的很是。倒是妹妹想的周到。因這些日子,為各處發貨,又為柳二哥的事忙了這幾日,把正經事都誤了。要不然,定了明兒後兒,下帖兒請罷。」薛姨媽道:「由你辦去罷。」

話猶未了,外面小廝在門外回說:「張管總着人送了兩個箱子來。」薛蟠聽了,便命小廝央門外幾個夥計搬進了兩個夾板夾的大棕箱。薛蟠一見說:「特給媽和妹妹帶來的東西,不是夥計送家裡來,我都忘了。」

薛姨媽同寶釵問:「是什麼好東西,這樣捆着夾着的?」便命人挑了繩子,去了夾板,開了鎖看時,卻是些綢緞、綾錦、洋貨等家常應用之物。獨有寶釵他的那個箱子裏,除了筆、墨、硯、各色箋紙、香袋、香珠、扇子、扇墜、花粉、胭脂、頭油等物外,還有虎丘帶來的自行人、酒令兒、水銀灌的打筋斗的小小子,沙子燈,一出一出的泥人兒的戲,用青紗罩的匣子裝着,又有在虎丘山上作的薛蟠的小像,泥捏成的與薛蟠毫無相差,以及許多碎小頑意兒的東西。寶釵一見,拿着薛蟠的小像細細看了,又看看他哥哥捂着嘴微笑,再和母兄說了一回閒話。便吩咐鶯兒:「你帶幾個老婆子,將我的這個箱子,拿到園子裏去,我好就近從那邊送人。」說着,便起身辭了母兄往園子裏去了。這裏薛姨媽將自己這個箱子裏的東西取出,一份一份的打點清楚,着鶯兒送往賈母並王夫人等處。

寶釵隨着箱子到了自己房中,將東西逐件過了目,除將自己留用之外,遂一一配妥當:也有送筆、墨、紙、硯的,也有送香袋、扇子、香墜的,也有送脂粉、頭油的,有單送頑意兒的。一一打點完畢,使鶯兒同一個老婆子,送往各處。

寶釵送東西的ㄚ頭回來,說:「也有道謝的,也有賞錢的,獨有給巧姐兒的那一份,仍舊拿回來了。」寶釵一見,不知何意,便問:「為什麼這一份沒送去,還是送了去沒收呢?」鶯兒說:「我方才給環哥兒送東西的時候,見璉二奶奶往老太太房裡去了。」寶釵說:「二奶奶不在家,你只管交給丫頭們收下,等二奶奶回來,自有他們告訴就是了。」鶯兒聽了,又與老婆子出了園子,到了鳳姐這邊,送了東西,回來見寶釵。

寶釵問道:「你見了璉二奶奶沒有?」鶯兒說:「我沒見。」寶釵說:「二奶奶還沒有回來?」鶯兒說:「回來是回來了。因豐兒對我說:『二奶奶自老太太屋裡回來,一臉怒氣,叫了平兒去,唧唧咕咕的說話,也不叫人聽見。你不必見,等我替你回一聲兒就是了。』因此豐兒拿進去,回了二奶奶。我們就回來了。」寶釵聽了,自己納悶,想不出鳳姐是為什麼生氣。

衆人不過收了東西,皆說些見面再謝等語而已。惟有林黛玉見是江南家鄉之物,便對着揮淚自嘆。紫鵑深知黛玉心腸,在一旁勸道:「寶姑娘送來這些東西,姑娘看着該喜歡才是。」

話猶未畢,只見寶玉已進來。寶玉見黛玉淚痕滿面,便問:「妹妹,又是為的什麼?」黛玉不答。旁邊紫鵑將嘴向牀後桌上一努,寶玉會意,便往牀上一看,見堆着許多東西,就知道是寶釵送來的。寶玉深知黛玉是因見了江南來的故鄉之物,勾起傷感落淚。便道:「妹妹,你放心!等我明年往江南去,與你帶兩船來。」黛玉聽了這話,說道:「你那裏知道我的緣故。」說着眼淚又流了下來。寶玉忙走到牀前,挨著黛玉坐下,將那些東西一件一件拿起來,擺弄著細瞧,故意問:「這是什麼,叫什麼名字?那是什麼做的,這樣齊整?這是什麼,要它做什麼使用?妹妹,你瞧,這一件可以擺在書閣兒上作陳設,那件放在條案上當古董兒倒好呢!」一味的將些沒要緊的話來支吾,搭訕。黛玉見寶玉可笑的樣子,稍將煩惱丟開。寶玉便說道:「寶姐姐送東西來給咱們,我想著,咱們也該到她那里道個謝去才是,不知妹妹可去不去?」黛玉道:「自家姐妹,這倒不必。只是到他那邊,薛大哥回來了,必然告訴他些南邊的新聞故事兒,我去聽聽,只當回了家鄉一趟的。」說著,眼圈兒又紅了。寶玉便站著等他。黛玉只得和他出來,往寶釵那裡去了。

二人到寶釵處,道了謝,寶玉又口口稱贊泥人兒等物有趣。寶釵笑道:「原不是什麼好東西,不過是遠路帶來的土物兒,大家看著新鮮些就是了。」黛玉道:「這些東西我們小時候倒不理會,如今看見,真是新鮮物兒了。」寶釵因笑道:「妹妹知道,這就是俗語說的『物離鄉貴』,其實可算什麼呢。」寶玉聽了這話正触著黛玉方纔的心事,連忙拿話岔開:「明年大哥哥還去江南嗎?——」話沒說完,黛玉早接口道:「——姐姐,你瞧,寶哥哥不是給姐姐來道謝,竟又要定下明年的東西來了。」說的寶釵寶玉都笑了。

三個人又閒話了一回,因提起黛玉的病來,寶釵勸了一回,因說道:「妹妹若覺著身上不爽快,倒要自己勉強扎掙著出來,各處走走逛逛,散散心,比在屋裡悶坐著到底好些。我那兩日,不是覺著發懶,渾身發熱,只是要歪著?也因為時氣不好,怕病,因此尋些事情,自己混著。這兩日才覺得好些了。」黛玉道:「姐姐說的何嘗不是?我也是這麼想著呢。」大家又坐了一會子方散。寶玉仍把黛玉送至瀟湘館門首,才各自回去了。

且說那趙姨娘因見寶釵送環哥兒物件,心中甚喜,滿嘴誇獎:「人人都說寶姑娘會行事,很大方,今日看來,果然不錯。他哥哥能帶了多少東西來,他挨家送到,並不遺漏一處,也不露出誰薄誰厚,連我們他都想到了,若是林姑娘,即或有人帶了東西來,那裏輪得到我們娘兒倆身上呢!可見人會行事,真真露着各別另樣的好。」趙姨娘因環哥兒得了東西,深為得意,不住的托在掌上擺弄瞧看一會。想寶釵乃係王夫人之表侄女,特要在王夫人跟前賣好兒。自己蝎蝎螫螫的拿着那東西,走至王夫人房中,站在一旁說道:「這是寶姑娘才給環哥的,他年輕輕的人想得周到,我還給了送東西的小ㄚ頭二百錢。聽說姨太太也給太太送來了,不知是什麼東西?你們瞧瞧這一個門裏頭,就是兩份兒,能有多少呢?怪不得老太太同太太都誇他疼他,果然招人疼。」說着,將手裏的東西遞過去與王夫人瞧,誰知王夫人頭也沒擡,手也沒伸,只口內說了聲「好,給環哥兒頑去罷」,並無正眼看一看。趙姨娘因招了一鼻子灰,滿肚氣惱,無精打彩的回房,將東西丟在一邊,也無人問他,他卻自己咕嘟着嘴,一邊子坐着。

且說薛蟠聽了母親之言,次日請了張德輝與四位夥計,俱已到齊,不免說些販賣賬目發貨之事。不一時,上席讓坐,薛蟠挨次斟了酒,薛姨媽又使人出來致謝,大家喝著酒說閒話兒。內中一個道:「今兒這席上短了柳二爺。」薛蟠聞言,把眉一皺,嘆口氣道:「什麼是柳二爺,如今不知那裡作『柳道爺』去了。」眾人都詫異道:「這是怎麼說?」薛蟠便把湘蓮前後事體說了一遍。眾人聽了,越發駭異,因說道:「怪不的。前兒我們在店裡,髣髣髴髴也聽見人吵嚷,說:『有一個道士,三言兩語,把一個人度了去了。』又說「『一陣風颳了去了。』只不知是誰。我們正發貨,那裡有閒工夫打聽這個事去?到如今還是似信不信的,誰知就是柳二爺呢?張德輝道:「柳二爺那樣個伶俐人,未必是真跟了道士去罷。他原會些武藝,又有力量,或看破那道士的妖術邪法,特意跟他去,在背地擺佈他,也未可知。」薛蟠道:「果然如此,倒也罷了。」眾夥計隨便喝了幾杯酒,吃了飯,大家散了。

話說寶玉回來,想着黛玉的孤苦,不免替他傷感起來。襲人見寶玉從外面進來坐在那發呆,便問:「就回來了?是不是同林姑娘一塊去了寶姑娘那兒?」寶玉說:「我會林姑娘同去的——送林姑娘的東西比送我們的多一兩倍呢。」說着話兒,便叫取了枕來,要在牀上歪着。襲人說:「璉二奶奶自從病了一場之後,我早就想着要到他那裏去看看,你同晴雯麝月呆着,我去看看就來。」寶玉說:「你只管去罷。」言畢,襲人遂換了兩件新鮮衣服。囑咐了晴雯、麝月幾句,便出了怡紅院。

至沁芳橋上立住,往四下里觀看那園中景致。那時正是夏末秋初,園內蟬鬧蟲鳴;只是花也開敗了,芙蓉池中荷葉新殘相間,也將殘上來了。倒是近着池邊,都發了紅鋪鋪的咕嘟子,襯着碧綠的葉兒,着實可愛。於是一壁里瞧着,一壁里下了橋。走了不遠,迎見李紈房裏的丫頭素雲捧着個洋漆盒兒走來。襲人便問:「往那裏去送東西?」素雲說:「這是我們奶奶給三姑娘送去的菱角兒、雞頭米。」襲人說:「這個東西,是咱們園子裏河內采的,還是外頭買來的呢?」素雲說:「是我們那邊劉媽媽的女兒從鄉下帶來孝敬我們奶奶的。因三姑娘在我們那裏坐,奶奶叫人剝了讓他吃。他說:『才吃了熱茶了,一會子再吃罷。』所以命我給三姑娘送過去。」言畢,各自散了。

襲人走著,沿堤看頑了一回。猛抬頭看見那邊葡萄架底下有人拿著撣子在那里撣什麼呢,走到跟前,卻是老祝媽。那老婆子見了襲人,便笑嘻嘻的迎上來,說道:「姑娘怎麼今日得工夫出來逛逛?」襲人道:「可不是。我要到璉二奶奶家去。你在這里做什麼呢?」那婆子道:「我在這里赶蜜蜂兒。今年三伏里雨水少,這果子樹上都有虫子,把果子吃的疤瘌流星的掉了好些下來。姑娘還不知道呢,這馬蜂最可惡的,一嘟嚕上只咬破三兩個兒,那破的水滴到好的上頭,連這一嘟嚕都是要爛的。姑娘你瞧,咱們說話的空兒沒赶,就落上許多了。」襲人道:「你就是不住手的赶,也赶不了許多。你倒是告訴買辦,叫他多多做些小冷布口袋兒,一嘟嚕套上一個,又透風,又不遭塌。」婆子笑道:「倒是姑娘說的是。我今年才管上,那里知道這個巧法兒呢。」

襲人說:「如今這園子裡這些果品有好些種,到是那樣先熟的快些?」老祝婆子說:「如今才入七月的門,果子都是才紅上來,要是好 吃,想來還得月盡頭兒才熟透了呢。姑娘不信,我摘一個給姑娘嚐嚐。」襲人正色說道:「這那裡使得?不但沒熟吃不得,就是熟了,一則沒有 供鮮,二則主子們尚然沒吃,我如何先吃得呢?」老婆子忙笑道:「姑娘說得有理。我因為姑娘問我,我白這樣說。」襲人說:“我方才告訴你要口袋的話,你就回一回二奶奶,叫管事的作去罷。」言畢,遂一直的出了園子的門,就到鳳姐這裡來了。

 一到院里,只聽鳳姐說道:「天理良心,我在這屋里熬的越發成了賊了。」襲人聽見這話,知道有原故了,又不好回來,又不好進去,遂把腳步放重些,隔著窗子問道:「平姐姐在家里呢么?」平兒忙答應著迎出來。襲人便問:「二奶奶也在家里呢么,身上可大安了?」說著,已走進來。鳳姐裝著在床上歪著呢,見襲人進來,也笑著站起來,說:「好些了,叫你惦著。怎麼這幾日不過我們這邊坐坐?」襲人道:「奶奶身上欠安,本該天天過來請安才是。但只怕奶奶身上不爽快,倒要靜靜兒的歇歇兒,我們來了,倒吵的奶奶煩。」鳳姐笑道:「常聽見平兒說你背地里還惦著我,常常問我。這就是你盡心了。」一面說著,叫平兒挪了張杌子放在床旁邊,讓襲人坐下。丰兒端進茶來,襲人欠身道:「妹妹坐著罷。」一面說閒話兒。只見一個小丫頭子在外間屋里悄悄的和平兒說:「旺兒來了。在二門上伺候著呢。」襲人知他們有事,又說了兩句話,便起身要走。鳳姐道:「閒來坐坐,說說話兒,我倒開心。」因命平兒:「送送你妹妹。」平兒答應著送出來。只見兩三個小丫頭子,都在那里屏聲息氣齊齊的伺候著。襲人不知何事,便自去了。

卻說平兒送出襲人,進來回道:「旺兒才來了,因襲人在這裡我叫他先到外頭等等兒,這會子還是立刻叫他呢,還是等著?請奶奶的示下。」鳳姐道:「叫他來。」平兒忙叫小丫頭去傳旺兒進來。這裡鳳姐又問平兒:「你到底是怎麼聽見說的?」平兒道:「就是頭裡那小丫頭子的話。他說他在二門裡頭聽見外頭兩個小廝說:『這個新二奶奶比咱們舊二奶奶還俊呢,脾氣兒也好。』不知是旺兒還是誰,吆喝了兩個一頓,說:『什麼新奶奶舊奶奶的,還不快悄悄兒的呢,叫裡頭知道了,把你的舌頭還割了呢。』」平兒正說著,只見一個小丫頭進來回說:「旺兒在外頭伺候著呢。」鳳姐聽了,冷笑了一聲說:「叫他進來。」那小丫頭出來說:「奶奶叫呢。」旺兒連忙答應著進來。旺兒請了安,在外間門口垂手侍立。鳳姐兒道:「你過來,我問你話。」旺兒才走到裡間門旁站著。鳳姐兒道:「你二爺在外頭弄了人,你知道不知道?」旺兒又打著千兒回道:「奴才天天在二門上聽差事,如何能知道二爺外頭的事呢。」鳳姐冷笑道:「你自然不知道。你要知道,你怎麼攔人呢。」旺兒見這話,知道剛纔的話已經走了風了,料著瞞不過,便又跪回道:「奴才實在不知。就是頭裡興兒和喜兒兩個人在那裡混說,奴才吆喝了他們兩句。內中深情底裡奴才不知道,不敢妄回。求奶奶問興兒,他是長跟二爺出門的。」鳳姐聽了,下死勁啐了一口,罵道:「你們這一起沒良心的混帳忘八崽子!都是一條藤兒,打量我不知道呢。先去給我把興兒那個忘八崽子叫了來,你也不許走。問明白了他,回來再問你。好,好,好,這才是我使出來的好人呢!」那旺兒只得連聲答應幾個是,磕了個頭爬起來出去,去叫興兒。

卻說興兒正在帳房兒里和小廝們玩呢,聽見說二奶奶叫,先唬了一跳,卻也想不到是這件事發作了,連忙跟著旺兒進來。旺兒先進去,回說:「興兒來了。」鳳姐兒厲聲道:「叫他!」那興兒聽見這個聲音兒,早已沒了主意了,只得乍著膽子進來。鳳姐兒一見,便說:「好小子啊!你和你爺辦的好事啊!你只實說罷!」興兒一聞此言,又看見鳳姐氣色,早唬軟了,不覺跪下,只是磕頭。鳳姐兒道:「論起這事來,我也聽見說不與你相干。但只你不早來回我知道,這就是你的不是了。你要實說了,我還饒你;再有一字虛言,你先摸摸你腔子上幾個腦袋瓜子!」興兒戰戰兢兢的朝上磕頭道:「奶奶問的是什麼事,奴才同爺辦壞了?」鳳姐聽了,一腔火都發作起來,喝命:「打嘴巴!」旺兒過來才要打時,鳳姐兒罵道:「什麼糊塗忘八崽子!叫他自己打,用你打嗎!一會子你再各人打你那嘴巴子還不遲呢。」那興兒真個自己左右開弓打了自己十幾個嘴巴。鳳姐兒喝聲「站住」,問道:「你二爺外頭娶了什麼新奶奶舊奶奶的事,你大概不知道啊。」興兒見說出這件事來,越發著了慌,連忙把帽子抓下來在磚地上咕咚咕咚碰的頭山響,口裡說道:「只求奶奶超生,奴才再不敢撒一個字兒的謊。」鳳姐道:「快說!」興兒直蹶蹶的跪起來回道:「這事頭裡奴才也不知道。就是這一天,東府里大老爺送了殯,俞祿往珍大爺廟裡去領銀子。二爺同著蓉哥兒到了東府里,道兒上爺兒兩個說起珍大奶奶那邊的二位姨奶奶來。二爺誇他好,蓉哥兒哄著二爺,說把二姨奶奶說給二爺。」鳳姐聽到這裡,使勁啐道:「呸,沒臉的忘八蛋!他是你那一門子的姨奶奶!」興兒忙又磕頭說:「奴才該死!」往上啾著,不敢言語。鳳姐兒道:「完了嗎?怎麼不說了?」興兒方纔又回道:「奶奶恕奴才,奴才才敢回。」鳳姐啐道:「放你媽的屁,這還什麼恕不恕了。你好生給我往下說,好多著呢。」 興兒又回道:「二爺聽見這個話就喜歡了。後來奴才也不知道怎麼就弄真了。」鳳姐微微冷笑道:「這個自然麽,你可那裡知道呢!你知道的只怕都煩了呢。是了,說底下的罷!」興兒回道:「後來就是蓉哥兒給二爺找了房子。」鳳姐忙問道:「如今房子在那裡?」興兒道:「就在府後頭。」鳳姐兒道:「哦。」回頭瞅著平兒道:「咱們都是死人哪。你聽聽!」平兒也不敢作聲。興兒又回道:「珍大爺那邊給了張家不知多少銀子,那張家就不問了。」鳳姐道:「這裡頭怎麼又扯拉上什麼張家李家咧呢?」興兒回道:「奶奶不知道,這二奶奶……」剛說到這裡,又自己打了個嘴巴,想了想,說道: 「那珍大奶奶的妹子……」鳳姐兒接著道:「怎麼樣?快說呀。」興兒道:「那珍大奶奶的妹子原來從小兒有人家的,姓張,叫什麼張華,如今窮的待好討飯。珍大爺許了他銀子,他就退了親了。」鳳姐兒聽到這裡,點了點頭兒,回頭便望平兒說道:「你都聽見了?小忘八崽子,頭裡他還說他不知道呢!」興兒又回道: 「後來二爺才叫人裱糊了房子,娶過來了。」鳳姐道:「打那裡娶過來的?」興兒回道:「就在他老娘家抬過來的。」鳳姐又問:「沒人送親麽?」興兒道:「就是蓉哥兒。還有幾個丫頭老婆子們,沒別人。」鳳姐道:「你大奶奶沒來嗎?」興兒道:「過了兩天,大奶奶才拿了些東西來瞧的。」鳳姐兒回頭向平兒道:「怪道那兩天二爺稱贊大奶奶不離嘴呢。」掉過臉來又問興兒,「誰伏侍呢?自然是你了。」興兒趕著碰頭不言語。鳳姐又問:「前頭那些日子說給那府里辦事,想來辦的就是這個了。」興兒回道:「也有辦事的時候,也有往新房子里去的時候。」

鳳姐聽了這一篇言詞,只氣得癡呆了半天,面如金紙,兩隻吊稍丹鳳眼越發直豎起來了,渾身亂戰。半晌,連話也說不上來,只是發怔。猛低頭,見興兒還在地下跪着,便說道:「你這個猴兒崽子就該打死。這有什麼瞞著我的?你想著瞞了我,就在你那糊塗爺跟前討了好兒了,你新奶奶好疼你。」興兒道:「未能早回奶奶,是奴才該死!」便叩頭有聲。

鳳姐又問道:「誰和他住著呢。」興兒道:「先是和他娘和妹子在一處。就在十幾天前,他妹子自己抹了脖子。他娘得病,昨兒也死了。」鳳姐道:「這又為什麼?」興兒隨將柳湘蓮的事說了一遍。鳳姐道:「這個人還算造化高,省了當那出名兒的忘八。」因又問道:「沒了別的事了麽?」興兒道:「別的事奴才不知道。奴才剛纔說的字字是實話,一字虛假,奶奶問出來只管打死奴才,奴才也無怨的。」鳳姐低了一回頭,便又指著興兒說道:「我不看你剛纔還有點怕懼兒,不敢撒謊,我把你的腿不給你砸折了。」說著喝聲「出去!」興兒瞌了個頭,才爬起來,退到外間門口,不敢就走。鳳姐道:「過來,我還有話呢。」興兒趕忙垂手敬聽。鳳姐道:「你忙什麼,新奶奶等著賞你什麼呢?」興兒也不敢抬頭。鳳姐道:「我什麼時候叫你,你什麼時候到。遲一步兒,你試試!出去罷。」興兒忙答應幾個「是」,退出門來。鳳姐又叫道:「興兒!」興兒趕忙站住。鳳姐道:「快出去告訴你二爺去,是不是啊?」興兒回道:「奴才不敢。」鳳姐道:「你出去提一個字兒,隄防你的皮!」興兒連忙答應著才出去了。鳳姐又叫:「旺兒呢?」旺兒連忙答應著過來。鳳姐把眼直瞪瞪的瞅了兩三句話的工夫,才說道:「好旺兒,很好,去罷!外頭有人提一個字兒,全在你身上。」旺兒答應著也出去了。

且說鳳姐見興兒出去,回頭向平兒說:「方才興兒說的話,你都聽見了沒有?天下那有這樣沒臉的男人!吃着碗裏,看着鍋里,見一個,愛一個,真成了餵不飽的狗,實在是個棄舊迎新的壞貨。只可惜這五六品的頂帶給他!他別想着俗語說的『家花那有野花香』的話,他要信了這個話,可就大錯了。多早晚在外面鬧一個沒臉、親戚朋友見不得的事出來,他才罷手呢!」平兒一旁勸道:「奶奶身子才好了,也不可過於氣惱。看二爺自從鮑二的女人那一件事之後,倒收了心,好了呢,如今為什麼又干起這樣事來?這都是珍大爺他的不是。」鳳姐說:「珍大爺固有不是,也總因咱們那位下作不堪的爺他眼饞,人家才引誘他的。俗語說『牛兒不吃水,也強按頭麼?』珍大爺幹這樣的事,珍大奶奶也該攔着不依才是。珍大奶奶也不想一想,把一個妹子要許幾家子弟才好,先許了姓張的,今又嫁了姓賈的;天下的男人都死絕了,都嫁到賈家來!難道賈家的衣食這樣好不成?那妹子本來也不是他親的,而且聽見說原是個混賬爛桃。難道珍大奶奶現做着命婦,家中有這樣一個打嘴現世的妹子,也不知道羞臊,躲避着些,反倒大面上揚鈴打鼓的,在這門裏丟醜,也不怕笑話?珍大爺也是做官的人,別的律例不知道也罷了,連個服中娶親,停妻再娶,使不得的規矩,他也不知道不成?你替他細想一想,他幹的這件事,是疼兄弟,還是害兄弟呢?」平兒說:「珍大爺只顧眼前,叫兄弟喜歡,也不管日後的輕重干係了。」鳳姐兒冷笑道:「這是什麼『叫兄弟喜歡』,這是給他毒藥吃!若論親叔伯兄弟中,他年紀又最大,又居長,不知教導兄弟學好,反引誘兄弟學不長進,擔罪名兒,日後鬧出事來,他在一邊缸沿兒上站着看熱鬧,真真我要罵也罵不出口來。他在那邊府里的醜事壞名聲,已經叫人聽不上了,必定也叫兄弟學他一樣,才好顯不出他的丑來。這是什麼作哥哥的道理?倒不如撒泡尿浸死了,替大老爺死了也罷,活着作什麼。」

平兒看鳳姐越說越氣,便跪在地下,再三苦勸安慰一會子,鳳姐才略消了些氣惱。喝了口茶,喘息一回,便要了拐枕,歪在牀上,閉眼養神。平兒只得悄悄的退出去了。鳳姐將前事從頭至尾細細的盤算多時,才得了主意,也不吿訴平兒,卻作出個嘻笑自若、毫無惱恨妒嫉的樣子來。心下早已算定,只待賈璉起程去平安州,再作道理。要知端的,且聽下回分解。
