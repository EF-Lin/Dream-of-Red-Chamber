\chapter{開生面夢演紅樓夢 立新場情傳幻境情}

題曰:

\begin{center}

春困葳蕤擁綉衾,恍隨仙子別紅塵。

問誰幻入華胥境,千古風流造孽人。

\end{center}

卻說薛家母子在榮府中寄居等事略已表明,此回則暫不能寫矣。

如今且說林黛玉自在榮府以來,賈母萬般憐愛,寢食起居,一如寶玉,迎春、探春、惜春三個親孫女倒且靠後。便是寶玉和黛玉二人之親密友愛處,亦自較別個不同,日則同行同坐,夜則同息同止,真是言和意順,略無參商。不想如今忽然來了一個薛寶釵,年歲雖大不多,然品格端方,容貌豐美,人多謂黛玉所不及。而且寶釵行為豁達,隨分從時,不比黛玉孤高自許,目無下塵,故比黛玉大得下人之心。便是那些小丫頭子們,亦多喜與寶釵去頑。因此黛玉心中便有些悒鬱不忿之意,寶釵卻渾然不覺。那寶玉亦在孩提之間,況自天性所稟來的一片愚拙偏僻,視姊妹弟兄皆出一意,並無親疏遠近之別。其中因與黛玉同隨賈母一處坐卧,故略比別個姊妹熟慣些。既熟慣,則更覺親密,既親密,則不免一時有求全之毀,不虞之隙。這日不知為何,他二人言語有些不合起來,黛玉又氣的獨在房中垂淚,寶玉又自悔言語冒撞,前去俯就,那黛玉方漸漸的迴轉來。

因東邊寧府中花園內梅花盛開,賈珍之妻尤氏乃治酒,請賈母、邢夫人、王夫人等賞花。是日先攜了賈蓉之妻,二人來面請。賈母等於早飯後過來,就在會芳園游頑,先茶後酒,不過皆是寧榮二府女眷家宴小集,並無別樣新文趣事可記。

一時寶玉倦怠,欲睡中覺,賈母命人好生哄著,歇一回再來。賈蓉之妻秦氏便忙笑回道:「我們這裡有給寶叔收拾下的屋子,老祖宗放心,只管交與我就是了。」又向寶玉的奶娘丫鬟等道:「嬤嬤姐姐們,請寶叔隨我這裡來。」賈母素知秦氏是個極妥當的人,生的裊娜纖巧,行事又溫柔和平,乃重孫媳中第一個得意之人,見他去安置寶玉,自是安穩的。

當下秦氏引了一簇人來至上房內間。寶玉抬頭看見一幅畫貼在上面,畫的人物固好,其故事乃是「燃藜圖」,也不看系何人所畫,心中便有些不快。又有一幅對聯,寫的是:

\begin{center}

世事洞明皆學問,人情練達即文章。

\end{center}
及看了這兩句,縱然室宇精美,鋪陳華麗,亦斷斷不肯在這裡了,忙說:「出去,出去!」秦氏聽了笑道:「這裡還不好,可往那裡去呢?不然往我屋裡去吧。」寶玉點頭微笑。有一個嬤嬤說道:「那裡有個叔叔往侄兒房裡睡覺的理?」秦氏笑道:「噯喲喲!不怕他惱。他能多大呢,就忌諱這些個!上月你沒看見我那個兄弟來了,雖然與寶叔同年,兩個人若站在一處,只怕那個還高些呢。」寶玉道:「我怎麼沒見過?你帶他來我瞧瞧。」眾人笑道:「隔著二三十里,往那裡帶去,見的日子有呢。」說著大家來至秦氏房中。剛至房門,便有一股細細的甜香襲人而來。寶玉覺得眼餳骨軟,連說: 「好香!」入房向壁上看時,有唐伯虎畫的《海棠春睡圖》,兩邊有宋學士秦太虛寫的一副對聯,其聯云:

\begin{center}

嫩寒鎖夢因春冷,芳氣籠人是酒香。

\end{center}

案上設著武則天當日鏡室中設的寶鏡,一邊擺著飛燕立著舞過的金盤,盤內盛著安祿山擲過傷了太真乳的木瓜。上面設著壽昌公主於含章殿下卧的榻,懸的是同昌公主制的聯珠帳。寶玉含笑連說:「這裡好!」秦氏笑道:「我這屋子大約神仙也可以住得了。」說著親自展開了西子浣過的紗衾,移了紅娘抱過的鴛枕,於是眾奶母伏侍寶玉卧好,款款散了,只留襲人、媚人、晴雯、麝月四個丫鬟為伴。秦氏便分咐小丫鬟們,好生在廊檐下看著貓兒狗兒打架。

那寶玉剛合上眼,便惚惚的睡去,猶似秦氏在前,遂悠悠蕩蕩,隨了秦氏,至一所在。但見朱欄白石,綠樹清溪,真是人跡希逢,飛塵不到。寶玉在夢中歡喜,想道:「這個去處有趣,我就在這裡過一生,縱然失了家也願意,強如天天被父母師傅打呢。」正胡思之間,忽聽山後有人作歌曰:

\begin{center}

春夢隨雲散,飛花逐水流。

寄言眾兒女,何必覓閑愁。

\end{center}

寶玉聽了是女子的聲音。歌聲未息,早見那邊走出一個人來,蹁躚裊娜,端的與人不同。有賦為證:

\begin{center}

方離柳塢,乍出花房。但行處,鳥驚庭樹;將到時,影度迴廊。仙袂乍飄兮,聞麝蘭之馥郁;荷衣欲動兮,聽環佩之鏗鏘。靨笑春桃兮,雲堆翠髻;唇綻櫻顆兮,榴齒含香。纖腰之楚楚兮,迴風舞雪;珠翠之輝輝兮,滿額鵝黃。出沒花間兮,宜嗔宜喜;徘徊池上兮,若飛若揚。蛾眉顰笑兮,將言而未語;蓮步乍移兮,待止而欲行。羡彼之良質兮,冰清玉潤;羡彼之華服兮,閃灼文章;愛彼之貌容兮,香培玉琢;美彼之態度兮,鳳翥龍翔。其素若何?春梅綻雪。其潔若何?秋菊被霜。其靜若何?松生空谷。其艷若何?霞映澄塘。其文若何?龍游曲沼。其神若何?月射寒江。應慚西子,實愧王嬙。奇矣哉,生於孰地,來自何方?信矣乎,瑤池不二,紫府無雙。果何人哉?如斯之美也!

\end{center}

寶玉見是一個仙姑,喜的忙來作揖問道:「神仙姐姐,不知從那裡來,如今要往那裡去?也不知這是何處,望乞攜帶攜帶。」那仙姑笑道:「吾居離恨天之上,灌愁海之中,乃放春山遣香洞太虛幻境警幻仙姑是也。司人間之風情月債,掌塵世之女怨男痴。因近來風流冤孽,纏綿於此處,是以前來訪察機會,布散相思。今忽與爾相逢,亦非偶然。此離吾境不遠,別無他物,僅有自採仙茗一盞,親釀美酒一瓮,素練魔舞歌姬數人,新填《紅樓夢》仙曲十二支,試隨吾一游否?」寶玉聽說,便忘了秦氏在何處,竟隨了仙姑,至一所在,有石牌橫建,上書「太虛幻境」四個大字,兩邊一副對聯,乃是:

\begin{center}

假作真時真亦假,無為有處有還無。

\end{center}

轉過牌坊,便是一座宮門,上面橫書四個大字,道是「孽海情天」。又有一副對聯,大書云:

\begin{center}

厚地高天堪嘆古今情不盡 痴男怨女可憐風月債難償

\end{center}

寶玉看了,心下自思道:「原來如此。但不知何為古今之情,何為風月之債?從今倒要領略領略。」寶玉只顧如此一想,不料早把些邪魔招入膏肓了。當下隨了仙姑進入二層門內,至兩邊配殿,皆有匾額對聯,一時看不盡許多,惟見有幾處寫的是:「痴情司」、「結怨司」、「朝啼司」、「夜怨司」、「春感司」、「秋悲司」。看了,因向仙姑道:「敢煩仙姑引我到那各司中游玩游玩,不知可使得?」仙姑道:「此各司中皆貯的是普天之下所有的女子過去未來的簿冊。爾凡眼塵軀,未便先知的。」寶玉聽了,那裡肯依,復央之再四。仙姑無奈,說: 「也罷,就在此司內略隨喜隨喜罷了。」寶玉喜不自勝,抬頭看這司的匾上,乃是「薄命司」三字,兩邊對聯寫的是:

\begin{center}

春恨秋悲皆自惹,花容月貌為誰妍。

\end{center}

寶玉看了,便知感嘆。進入門來,只見有數十個大廚,皆用封條封着。看那封條上,皆是各省地名。寶玉一心只揀自己的家鄉的封條看,遂無心看別省的了。只見那邊廚上封條上大書七字云:金陵十二釵正冊。寶玉因問:「何為金陵十二釵正冊?」警幻道:「即貴省中十二冠首女子之冊,故為正冊。」寶玉道:「常聽人說,金陵極大,怎麼只十二個女子?如今單我們家裡,上上下下就有幾百女孩子呢。」警幻冷笑道:「貴省女子固多,不過擇其緊要者錄之。下邊二廚則又次之。餘者庸愚之輩,則無冊可錄矣。」寶玉聽說,再看下首二廚上,果然一個寫着金陵十二釵副冊,又一個寫着金陵十二釵又副冊。寶玉便伸手先將又副冊廚開了,拿出一本冊來,揭開一看,只見這首頁上畫着一副畫,又非人物,亦無山水,不過水墨滃染的滿紙烏雲濁霧而已。後有幾行字,寫的是:

\begin{center}

霽月難逢,彩雲易散。

心比天高,身為下賤。

風流靈巧招人怨。

壽夭多因誹謗生,

多情公子空牽念。

\end{center}

寶玉看了,又見後面畫着一簇鮮花,一床破席。也有幾句言詞,寫道是:

\begin{center}

枉自溫柔和順,空云似桂如蘭。

堪羡優伶有福,誰知公子無緣。

\end{center}

寶玉看了不解。遂擲下這個,又去開了副冊廚門,拿起一本冊來,揭開看時,只見畫着一株桂花,下面有一池沼,其中水涸泥乾,蓮枯藕敗。畫後書云:

\begin{center}

根並荷花一莖香,平生遭際實堪傷。

自從兩地生孤木,致使香魂返故鄉。

\end{center}

寶玉看了仍不解他。又擲下,再去取正冊看。只見頭一頁上便畫着兩株枯木,木上懸着一圍玉帶,又有一堆雪,雪下一股金簪。也有四句言詞道:

\begin{center}

可嘆停機德,堪憐咏絮才。

玉帶林中掛,金簪雪裡埋。

\end{center}

寶玉看了仍不解。待要問時,情知他必不肯泄漏;待要丟下,又不舍。遂又往後看時,只見畫著一張弓,弓上掛一香櫞。也有一首歌詞云:

\begin{center}

二十年來辨是誰,榴花開處照宮闈;

三春爭及初春景,虎兔相逢大夢歸。

\end{center}

\footnote{按:亦寫作「虎兕相逢大夢歸」}

後面又畫著兩人放風箏,一片大海,一隻大船,船中有一女子掩面泣涕之狀。也有四句寫云:

\begin{center}

才自精明志自高,生於末世運偏消。

清明涕送江邊望,千里東風一夢遙。

\end{center}

後面又畫幾縷飛雲,一灣逝水。其詞曰:

\begin{center}

富貴又何為?襁褓之間父母違;

展眼弔斜暉,湘江水逝楚雲飛。

\end{center}

後面又畫著一塊美玉,落在泥垢之中。其斷語云:

\begin{center}

欲潔何曾潔,云空未必空!

可憐金玉質,落陷污泥中。

\end{center}

後面忽見畫著個惡狼,追撲一美女,欲啖之意。其書云:

\begin{center}

子系中山狼,得志便猖狂。

金閨花柳質,一載赴黃梁。

\end{center}

後面便是一所古廟,裡面有一美人在內看經獨坐。其判云:

\begin{center}

勘破三春景不長,緇衣頓改昔年妝。

可憐綉戶侯門女,獨卧青燈古佛傍。

\end{center}

後面便是一片冰山,上面有一隻雌鳳。其判曰:

\begin{center}

凡鳥偏從末世來,都知愛慕此身才。

一從二令三人木,哭向金陵事更哀。

\end{center}

後面又是一座荒村野店,有一美人在那裡紡績。其判云:

\begin{center}

事敗休云貴,家亡莫論親。

偶因濟劉氏,巧得遇恩人。

\end{center}

後面又畫著一盆茂蘭,旁有一位鳳冠霞帔的美人。也有判云:

\begin{center}

桃李春風結子完,到頭誰似一盆蘭?

為冰為水空相妒,枉與他人作話談。

\end{center}

後面又畫著高樓大廈,有一美人懸梁自縊。其判云:

\begin{center}

情天情海幻情身,情既相逢必主淫。

謾言不肖皆榮出,造釁開端實在寧。

\end{center}

寶玉還欲看時,那仙姑知他天分高明,性情穎慧,恐把仙機泄漏,遂掩了捲冊,笑向寶玉道:「且隨我去游玩奇景,何必在此打這悶葫蘆!」

寶玉恍恍惚惚,不覺棄了捲冊,又隨了警幻來至後面。但見珠簾繡幕,畫棟雕檐,說不盡那光搖朱戶金鋪地,雪照瓊窗玉作宮。更見仙花馥郁,異草芬芳,真好個所在。又聽警幻笑道:「你們快出來迎接貴客!」一語未了,只見房中又走出幾個仙子來,皆是荷袂蹁躚,羽衣飄舞,姣若春花,媚如秋月。一見了寶玉,都怨謗警幻道:「我們不知系何『貴客』,忙的接了出來!姐姐曾說今日今時必有絳珠妹子的生魂前來游玩,故我等久待。何故反引這濁物來污染這清凈女兒之境?」寶玉聽如此說,便嚇得欲退不能退,果覺自形污穢不堪。警幻忙攜住寶玉的手,向眾姊妹道:「你等不知原委:今日原欲往榮府去接絳珠,適從寧府所過,偶遇寧榮二公之靈,囑吾云:『吾家自國朝定鼎以來,功名奕世,富貴傳流,雖歷百年,奈運終數盡,不可輓回者。故遺之子孫雖多,竟無可以繼業。其中惟嫡孫寶玉一人,稟性乖張,生性怪譎,雖聰明靈慧,略可望成,無奈吾家運數合終,恐無人規引入正。幸仙姑偶來,萬望先以情欲聲色等事警其痴頑,或能使彼跳出迷人圈子,然後入於正路,亦吾兄弟之幸矣。』如此囑吾,故發慈心,引彼至此。先以彼家上中下三等女子之終身冊籍,令彼熟玩,尚未覺悟。故引彼再至此處,令其再歷飲饌聲色之幻,或冀將來一悟,亦未可知也。」

說畢,攜了寶玉入室。但聞一縷幽香,竟不知其所焚何物。寶玉遂不禁相問,警幻冷笑道:「此香塵世中既無,爾何能知!此香乃系諸名山勝境內初生異卉之精,合各種寶林珠樹之油所制,名『群芳髓』。」寶玉聽了,自是羡慕而已。大家入座,小丫鬟捧上茶來。寶玉自覺清香異味,純美非常,因又問何名。警幻道:「此茶出在放春山遣香洞,又以仙花靈葉上所帶之宿露而烹。此茶名曰『千紅一窟』。」寶玉聽了,點頭稱賞。因看房內,瑤琴、寶鼎、古畫、新詩,無所不有,更喜窗下亦有唾絨,奩間時漬粉污。壁上也見懸著一副對聯,書云:

幽微靈秀地, 無可奈何天。

寶玉看畢,無不羡慕。因又請問眾仙姑姓名:一名痴夢仙姑,一名鐘情大士,一名引愁金女,一名度恨菩提,各各道號不一。少刻,有小丫鬟來調桌安椅,設擺酒饌。真是:瓊漿滿泛玻璃盞,玉液濃斟琥珀杯。更不用再說那餚饌之盛。寶玉因聞得此酒清香甘冽,異乎尋常,又不禁相問。警幻道:「此酒乃以百花之蕊,萬木之汁,加以麟髓之醅,鳳乳之麯釀成,因名為『萬艷同杯』。」寶玉稱賞不迭。

飲酒間,又有十二個舞女上來,請問演何詞曲。警幻道:「就將新制《紅樓夢》十二支演上來。」舞女們答應了,便輕敲檀板,款按銀箏。聽他歌道是:

\begin{center}

開闢鴻濛……

\end{center}

方歌了一句,警幻便說道:「此曲不比塵世中所填傳奇之曲,必有生旦凈末之則,又有南北九宮之限。此或詠嘆一人,或感懷一事,偶成一曲,即可譜入管弦。若非個中人,不知其中之妙。料爾亦未必深明此調,若不先閱其稿,後聽其歌,翻成嚼蠟矣。」說畢,回頭命小丫鬟取了《紅樓夢》原稿來,遞與寶玉。寶玉接來,一面目視其文,一面耳聆其歌曰:

\begin{center}

第一支,紅樓夢引子:

\end{center}

開闢鴻濛,誰為情種?都只為風月情濃。趁着這奈何天、傷懷日、寂寞時,試遣愚衷。因此上,演出這懷金悼玉的《紅樓夢》。

\begin{center}

第二支,終身悞:

\end{center}

都道是金玉良姻,俺只念木石前盟。空對著,山中高士晶瑩雪;終不忘,世外仙姝寂寞林。嘆人間,美中不足今方信。縱然是齊眉舉案,到底意難平。

\begin{center}

第三支,枉凝眉:

\end{center}

一個是閬苑仙葩,一個是美玉無瑕。若說沒奇緣,今生偏又遇着他;若說有奇緣,如何心事終須化!一個枉自嗟呀,一個空勞牽掛。一個是水中月,一個是鏡中花。想眼中,能有多少淚珠兒,怎經得,秋流到冬盡春流到夏。\newline

寶玉聽了此曲,散漫無稽,不見得好處,但其聲韻凄惋,竟能銷魂醉魄。因此也不察其原委,問其來歷,就暫以此釋悶而已。因又看下道:

\begin{center}

第四支,恨無常:

\end{center}

喜榮華正好,恨無常又到。眼睜睜,把萬事全拋;盪悠悠,把芳魂消耗。望家鄉,路遠山遙。故向爹娘夢裡相尋告:兒命已入黃泉,天倫呵,須要退步抽身早。

\begin{center}

第五支,分骨肉:

\end{center}

一帆風雨路三千,把骨肉家園齊來拋閃。恐哭損殘年。告爹娘,莫把兒懸念。自古窮通皆有命,離合豈無緣。從今分兩地,各自保平安。奴去也,莫牽連。

\begin{center}

第六支,樂中悲:

\end{center}

襁褓中,父母嘆雙亡。縱居那綺羅叢,誰知嬌養?幸生來,英雄闊大寬宏量,從未將兒女私情略縈心上。好一似,霽月光風耀玉堂。廝配得才貌仙郎,博得個地久天長,準折得幼年時坎坷形狀。終久是雲散高唐,水涸湘江。這是塵寰中消長數應當,何必枉悲傷!

\begin{center}

第七支,世難容:

\end{center}

氣質美如蘭,才華阜比仙。天生成孤癖人皆罕。你道是啖肉食腥膻,視綺羅俗厭;卻不知太高人愈妒,過潔世同嫌。可嘆這,青燈古殿人將老;辜負了,紅粉朱樓春色闌。到頭來,依舊是風塵骯髒違心愿;好一似,無瑕美玉遭泥陷。又何須,王孫公子嘆無緣。

\begin{center}

第八支,喜寃家:

\end{center}

中山狼,無情獸,全不念當日根由。一味的,驕奢淫蕩貪還構。覷著那,侯門艷質同蒲柳;作踐的,公府千金似下流。嘆芳魂艷魄,一載盪悠悠。

\begin{center}

第九支,虛花悟:

\end{center}

將那三春看破,桃紅柳綠待如何?把這韶華打滅,覓那情淡天和。說什麼,天上夭桃盛,雲中杏蕊多!到頭來,誰見把秋捱過?則看那,白楊村裡人嗚咽,青楓林下鬼吟哦。更兼着,連天衰草遮墳墓。這的是,昨貧今富人勞碌,春榮秋謝花折磨。似這般,生關死劫誰能躲?聞道說,西方寶樹喚婆娑,上結著長生果。

\begin{center}

第十支,聰明累:

\end{center}

機關算盡太聰明,反算了卿卿性命。生前心已碎,死後性靈空。家富人寧,終有個,家亡人散各奔騰。枉費了,意憖憖半世心;好一似,盪悠悠三更夢。忽喇喇如大廈傾,昏慘慘似燈將盡。呀!一場歡喜忽悲辛。嘆人世,終難定!

\begin{center}

第十一支,留餘慶:

\end{center}

留餘慶,留餘慶,忽遇恩人;幸娘親,幸娘親,積得陰功。勸人生,濟困扶窮,休似俺那銀錢上,忘骨肉的狠舅奸兄!正是乘除加減,上有蒼穹。

\begin{center}

第十二支,晚韶華:

\end{center}

鏡裡恩情,更那堪夢裡功名!那美韶華去之何迅!再休提綉帳鴛衾。只這戴珠冠,披鳳襖,也抵不了無常性命。雖說是,人生莫受老來貧,也須要陰騭積兒孫。氣昂昂頭戴簪纓,光閃閃腰懸金印;威赫赫爵位高登,昏慘慘黃泉路近。問古來將相可還存?也只是虛名兒與後人欽敬。

\begin{center}

第十三支,好事終:

\end{center}

畫梁春盡落香塵。擅風情,秉月貌,便是敗家的根本。箕裘頹墮皆從敬,家事消亡首罪寧。宿孽總因情。

\begin{center}

第十四支,收尾·飛鳥各投林:

\end{center}

為官的,家業凋零;富貴的,金銀散盡。有恩的,死裡逃生;無情的,分明照應。欠命的,命已還;欠淚的,淚已盡。冤冤相報實非輕,分離合聚皆前定。欲知命短問前生,老來富貴也真僥幸。看破的,遁入空門;痴迷的,枉送了性命。好一似食盡鳥投林,落了片白茫茫大地真乾凈!

歌畢,還又歌別曲。警幻見寶玉甚無趣味,因嘆:「痴兒竟尚未悟!」那寶玉忙止歌姬不必再曲,自覺朦朧恍惚,告醉求卧。警幻便命撤去殘席,送寶玉至一香閨繡閣之中,其間鋪陳之盛,乃素所未見之物。更可駭者,早有一位女子在內,其鮮艷嫵媚,有似乎寶釵,風流裊娜,則又如黛玉。正不知何意。忽警幻道:「塵世中多少富貴之家,那些綠窗風月,繡閣煙霞,皆被淫污紈絝與那些流蕩女子悉皆玷辱。更可恨者,自古來多少輕薄浪子,皆以好色不淫為飾,又以情而不淫作案,此皆飾非掩醜之語也。好色即淫,知情更淫。是以巫山之會,雲雨之歡,皆由既悅其色,復戀其情所致也。吾所愛汝者,乃天下古今第一淫人也。」 寶玉聽了,唬的忙答道:「仙姑差了。我因懶於讀書,家父母尚每垂訓飭,豈敢再冒淫字?況且年紀尚小。不知淫字為何物。」警幻道:「非也。淫雖一理。意則有別。如世之好淫者,不過悅容貌,喜歌舞,調笑無厭,雲雨無時,恨不能盡天下之美女供我片時之趣興,此皆皮膚淫濫之蠢物耳。如爾則天分中生成一段痴情,吾輩推之為『意淫』。『意淫』二字,惟心會而不可口傳,可神通而不能語達。汝今獨得此二字,在閨闥中,固可為良友,然於世道中未免迂闊怪詭,百口嘲謗,萬目睚眦。今既遇令祖寧榮二公剖腹深囑,吾不忍君獨為我閨閣增光,見棄於世道,是特引前來,醉以靈酒,沁以仙茗,警以妙曲,再將吾妹一人,乳名兼美字可卿者,許配於汝。今夕良時,即可成姻。不過領汝領略此仙閨幻境之風光,尚然如此,何況塵境之情哉?今而後萬萬解釋,改悟前情,将謹勤有用的工夫,置身於經濟之道。」說畢便秘授以雲雨之事,推寶玉入帳。那寶玉恍恍惚惚,依警幻所囑之言,未免有陽臺巫峽之會。数日来,柔情綣繾,軟語溫存,與可卿難解難分。

那日,警幻携寶玉、可卿閒遊至一個所在,但見荊榛遍地,狼虎同群,忽爾大河阻路,黑水淌洋,又無橋梁可通。寶玉正自徬徨,只聽警幻道:「寶玉再休前進,作速回頭要緊!」寶玉忙止步問道:「此系何處?」警幻道:「此即迷津也。深有萬丈,遙亘千里,中無舟楫可通,只有一個木筏,乃木居士掌舵,灰侍者撐篙,不受金銀之謝,但遇有緣者渡之。爾今偶游至此,如墮落其中,則深負我從前一番以情悟道、守理衷情之言。”寶玉方欲回言,只聽迷津內水響如雷,竟有一夜叉般怪物攛出,直撲而來。嚇得寶玉汗下如雨,一面失聲喊叫:「可卿救我!可卿救我!」慌得襲人、媚人等上來扶起,拉手說:「寶玉別怕,我們在這裡!」秦氏在外聽見,連忙進來,一面說ㄚ鬟們好生看着貓兒狗兒打架,又聞寶玉口中连叫可卿救我,因納悶道:「我的小名,這裡沒人知道,他如何從夢裡叫出來?」正是:

\begin{center}

一場幽夢同誰訴,千古情人獨我知。

\end{center}
