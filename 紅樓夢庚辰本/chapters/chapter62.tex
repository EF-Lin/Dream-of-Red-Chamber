\chapter{憨湘雲醉眠芍藥裀 呆香菱情解石榴裙}
那柳家的笑道:「好猴兒崽子,你親嬸子找野老兒去了,你豈不多得一個叔叔,有什麼疑的!別討我把你頭上的榪子蓋似的幾根屄毛撏下來!還不開門讓我進去呢。」這小廝且不開門,且拉著笑說:「好嬸子,你這一進去,好歹偷些杏子出來賞我吃。我這裡老等。你若忘了時,日後半夜三更打酒買油的,我不給你老人家開門,也不答應你,隨你乾叫去。」柳氏啐道:「發了昏的,今年不比往年,把這些東西都分給了眾奶奶了。一個個的不象抓破了臉的,人打樹底下一過,兩眼就象那黧雞似的,還動他的果子!昨兒我從李子樹下一走,偏有一個蜜蜂兒往臉上一過,我一招手兒,偏你那好舅母就看見了。他離的遠看不真,只當我摘李子呢,就屄聲浪嗓喊起來,說又是『還沒供佛呢』,又是『老太太、太太不在家還沒進鮮呢,等進了上頭,嫂子們都有分的』,倒象誰害了饞癆等李子出汗呢。叫我也沒好話說,搶白了他一頓。可是你舅母姨娘兩三個親戚都管著,怎不和他們要的,倒和我來要。這可是『倉老鼠和老鴰去借糧——守著的沒有,飛著的有』。」小廝笑道:「哎喲喲,沒有罷了,說上這些閑話!我看你老以後就用不著我了?就便是姐姐有了好地方,將來更呼喚著的日子多,只要我們多答應他些就有了。」柳氏聽了,笑道: 「你這個小猴精,又搗鬼弔白的,你姐姐有什麼好地方了?」那小廝笑道:「別哄我了,早已知道了。單是你們有內牽,難道我們就沒有內牽不成?我雖在這裡聽哈,裡頭卻也有兩個姊妹成個體統的,什麼事瞞了我們!」

正說著,只聽門內又有老婆子向外叫:「小猴兒們,快傳你柳嬸子去罷,再不來可就誤了。」柳家的聽了,不顧和小廝說話,忙推門進去,笑說:「不必忙,我來了。」一面來至廚房,——雖有幾個同伴的人,他們都不敢自專,單等他來調停分派——一面問眾人:「五丫頭那去了?」眾人都說:「才往茶房裡找他們姊妹去了。」

柳家的聽了,便將茯苓霜擱起,且按著房頭分派菜饌。忽見迎春房裡小丫頭蓮花兒走來說:「司棋姐姐說了,要碗雞蛋,燉的嫩嫩的。」柳家的道:「就是這樣尊貴。不知怎的,今年這雞蛋短的很,十個錢一個還找不出來。昨兒上頭給親戚家送粥米去,四五個買辦出去,好容易才湊了二千個來。我那裡找去?你說給他,改日吃罷。」蓮花兒道:「前兒要吃豆腐,你弄了些餿的,叫他說了我一頓。今兒要雞蛋又沒有了。什麼好東西,我就不信連雞蛋都沒有了,別叫我翻出來。」一面說,一面真個走來,揭起菜箱一看,只見裡面果有十來個雞蛋,說道:「這不是?你就這麼利害!吃的是主子的,我們的分例,你為什麼心疼?又不是你下的蛋,怕人吃了。」柳家的忙丟了手裡的活計,便上來說道:「你少滿嘴裡混唚!你娘才下蛋呢!通共留下這幾個,預備菜上的澆頭。姑娘們不要,還不肯做上去呢,預備接急的。你們吃了,倘或一聲要起來,沒有好的,連雞蛋都沒了。你們深宅大院,水來伸手,飯來張口,只知雞蛋是平常物件,那裡知道外頭買賣的行市呢。別說這個,有一年連草根子還沒了的日子還有呢。我勸他們,細米白飯,每日肥雞大鴨子,將就些兒也罷了。吃膩了膈,天天又鬧起故事來了。雞蛋、豆腐,又是什麼麵筋、醬蘿蔔炸兒,敢自倒換口味。只是我又不是答應你們的,一處要一樣,就是十來樣。我倒別伺候頭層主子,只預備你們二層主子了。」蓮花聽了,便紅了臉,喊道:「誰天天要你什麼來?你說上這兩車子話!叫你來,不是為便宜卻為什麼。前兒小燕來,說晴雯姐姐要吃蘆蒿,你怎麼忙的還問肉炒雞炒?小燕說:『葷的因不好才另叫你炒個麵筋的,少擱油才好。』你忙的倒說自己發昏,趕著洗手炒了,狗顛兒似的親捧了去。今兒反倒拿我作筏子,說我給眾人聽。」柳家的忙道:「阿彌陀佛!這些人眼見的。別說前兒一次,就從舊年一立廚房以來,凡各房裡偶然間不論姑娘姐兒們要添一樣半樣,誰不是先拿了錢來,另買另添。有的沒的,名聲好聽,說我單管姑娘廚房省事,又有剩頭兒,算起帳來,惹人噁心:連姑娘帶姐兒們四五十人,一日也只管要兩隻雞,兩隻鴨子,十來斤肉,一弔錢的菜蔬。你們算算,夠作什麼的?連本項兩頓飯還撐持不住,還擱的住這個點這樣,那個點那樣,買來的又不吃,又買別的去。既這樣,不如回了太太,多添些分例,也象大廚房裡預備老太太的飯,把天下所有的菜蔬用水牌寫了,天天轉著吃,吃到一個月現算倒好。連前兒三姑娘和寶姑娘偶然商議了要吃個油鹽炒枸杞芽兒來,現打發個姐兒拿著五百錢來給我,我倒笑起來了,說:『二位姑娘就是大肚子彌勒佛,也吃不了五百錢的去。這三二十個錢的事,還預備的起。』趕著我送回錢去,到底不收,說賞我打酒吃,又說:『如今廚房在裡頭,保不住屋裡的人不去叨登,一鹽一醬,那不是錢買的。你不給又不好,給了你又沒的賠。你拿著這個錢,全當還了他們素日叨登的東西窩兒。』這就是明白體下的姑娘,我們心裡只替他念佛。沒的趙姨奶奶聽了又氣不忿,又說太便宜了我,隔不了十天,也打發個小丫頭子來尋這樣尋那樣,我倒好笑起來。你們竟成了例,不是這個,就是那個,我那裡有這些賠的。」

正亂時,只見司棋又打發人來催蓮花兒,說他:「死在這裡了,怎麼就不回去?」蓮花兒賭氣回來,便添了一篇話,告訴了司棋。司棋聽了,不免心頭起火。此刻伺候迎春飯罷,帶了小丫頭們走來,見了許多人正吃飯,見他來的勢頭不好,都忙起身陪笑讓坐。司棋便喝命小丫頭子動手,「凡箱櫃所有的菜蔬,只管丟出來喂狗,大家賺不成。」小丫頭子們巴不得一聲,七手八腳搶上去,一頓亂翻亂擲的。眾人一面拉勸,一面央告司棋說:「姑娘別誤聽了小孩子的話。柳嫂子有八個頭,也不敢得罪姑娘,說雞蛋難買是真。我們才也說他不知好歹,憑是什麼東西,也少不得變法兒去。他已經悟過來了,連忙蒸上了。姑娘不信瞧那火上。」

司棋被眾人一頓好言,方將氣勸的漸平。小丫頭們也沒得摔完東西,便拉開了。司棋連說帶罵,鬧了一回,方被眾人勸去。柳家的只好摔碗丟盤自己咕嘟了一回,蒸了一碗蛋令人送去。司棋全潑了地下了。那人回來也不敢說,恐又生事。

柳家的打發他女兒喝了一回湯,吃了半碗粥,又將茯苓霜一節說了。五兒聽罷,便心下要分些贈芳官,遂用紙另包了一半,趁黃昏人稀之時,自己花遮柳隱的來找芳官。且喜無人盤問。一徑到了怡紅院門前,不好進去,只在一簇玫瑰花前站立,遠遠的望著。有一盞茶時,可巧小燕出來,忙上前叫住。小燕不知是那一個,至跟前方看真切,因問作什麼。五兒笑道:「你叫出芳官來,我和他說話。」小燕悄笑道:「姐姐太性急了,橫豎等十來日就來了,只管找他做什麼。方纔使了他往前頭去了,你且等他一等。不然,有什麼話告訴我,等我告訴他。恐怕你等不得,只怕關園門了。」五兒便將茯苓霜遞與了小燕,又說這是茯苓霜,如何吃,如何補益,「我得了些送他的,轉煩你遞與他就是了。」說畢,作辭回來。

正走蓼漵一帶,忽見迎頭林之孝家的帶著幾個婆子走來,五兒藏躲不及,只得上來問好。林之孝家的問道:「我聽見你病了,怎麼跑到這裡來?」五兒陪笑道: 「因這兩日好些,跟我媽進來散散悶。才因我媽使我到怡紅院送家伙去。」林之孝家的說道:「這話岔了。方纔我見你媽出來我才關門。既是你媽使了你去,他如何不告訴我說你在這裡呢,竟出去讓我關門,是何主意?可知是你扯謊。」五兒聽了,沒話回答,只說:「原是我媽一早教我取去的,我忘了,挨到這時我才想起來了。只怕我媽錯當我先出去了,所以沒和大娘說得。」

林之孝家的聽他辭鈍色虛,又因近日玉釧兒說那邊正房內失落了東西,幾個丫頭對賴,沒主兒,心下便起了疑。可巧小蟬、蓮花兒並幾個媳婦子走來,見了這事,便說道:「林奶奶倒要審審他。這兩日他往這裡頭跑的不象,鬼鬼唧唧的,不知幹些什麼事。」小蟬又道:「正是。昨兒玉釧姐姐說,太太耳房裡的柜子開了,少了好些零碎東西。璉二奶奶打發平姑娘和玉釧姐姐要些玫瑰露,誰知也少了一罐子。若不是尋露,還不知道呢。」蓮花兒笑道:「這話我沒聽見,今兒我倒看見一個露瓶子。」林之孝家的正因這些事沒主兒,每日鳳姐使平兒催逼他,一聽此言,忙問在那裡。蓮花兒便說:「在他們廚房裡呢。」林之孝家的聽了,忙命打了燈籠,帶著眾人來尋。五兒急的便說:「那原是寶二爺屋裡的芳官給我的。」林之孝家的便說:「不管你方官圓官,現有了贓證,我只呈報了,憑你主子前辯去。」一面說,一面進入廚房,蓮花兒帶著,取出露瓶。恐還有偷的別物,又細細搜了一遍,又得了一包茯苓霜,一併拿了,帶了五兒,來回李紈與探春。

那時李紈正因蘭哥兒病了,不理事務,只命去見探春。探春已歸房。人回進去,丫鬟們都在院內納涼,探春在內盥沐,只有待書回進去。半日,出來說:「姑娘知道了,叫你們找平兒回二奶奶去。」林之孝家的只得領出來。到鳳姐兒那邊,先找著了平兒,平兒進去回了鳳姐。鳳姐方纔歇下,聽見此事,便吩咐:「將他娘打四十板子,攆出去,永不許進二門。把五兒打四十板子,立刻交給莊子上,或賣或配人。」平兒聽了,出來依言吩咐了林之孝家的。五兒唬的哭哭啼啼,給平兒跪著,細訴芳官之事。平兒道:「這也不難,等明日問了芳官便知真假。但這茯苓霜前日人送了來,還等老太太、太太回來看了才敢打動,這不該偷了去。」五兒見問,忙又將他舅舅送的一節說了出來。平兒聽了,笑道:「這樣說,你竟是個平白無辜之人,拿你來頂缸。此時天晚,奶奶才進了藥歇下,不便為這點子小事去絮叨。如今且將他交給上夜的人看守一夜,等明兒我回了奶奶,再做道理。」林之孝家的不敢違拗,只得帶了出來交與上夜的媳婦們看守,自便去了。

這裡五兒被人軟禁起來,一步不敢多走。又兼眾媳婦也有勸他說,不該做這沒行止之事;也有報怨說,正經更還坐不上來,又弄個賊來給我們看,倘或眼不見尋了死,逃走了,都是我們不是。於是又有素日一干與柳家不睦的人,見了這般,十分趁願,都來奚落嘲戲他。這五兒心內又氣又委屈,竟無處可訴;且本來怯弱有病,這一夜思茶無茶,思水無水,思睡無衾枕,嗚嗚咽咽直哭了一夜。

誰知和他母女不和的那些人,巴不得一時攆出他們去,惟恐次日有變,大家先起了個清早,都悄悄的來買轉平兒,一面送些東西,一面又奉承他辦事簡斷,一面又講述他母親素日許多不好。平兒一一的都應著,打發他們去了,卻悄悄的來訪襲人,問他可果真芳官給他露了。襲人便說:「露卻是給芳官,芳官轉給何人我卻不知。」襲人於是又問芳官,芳官聽了,唬天跳地,忙應是自己送他的。芳官便又告訴了寶玉,寶玉也慌了,說:「露雖有了,若勾起茯苓霜來,他自然也實供。若聽見了是他舅舅門上得的,他舅舅又有了不是,豈不是人家的好意,反被咱們陷害了。」因忙和平兒計議:「露的事雖完,然這霜也是有不是的。好姐姐,你叫他說也是芳官給他的就完了。」平兒笑道:「雖如此,只是他昨晚已經同人說是他舅舅給的了,如何又說你給的?況且那邊所丟的露也是無主兒,如今有贓證的白放了,又去找誰?誰還肯認?眾人也未必心服。」晴雯走來笑道:「太太那邊的露再無別人,分明是彩雲偷了給環哥兒去了。你們可瞎亂說。」平兒笑道:「誰不知是這個原故,但今玉釧兒急的哭,悄悄問著他,他應了,玉釧也罷了,大家也就混著不問了。難道我們好意兜攬這事不成!可恨彩雲不但不應,他還擠玉釧兒,說他偷了去了。兩個人窩裡發炮,先吵的合府皆知,我們如何裝沒事人。少不得要查的。殊不知告失盜的就是賊,又沒贓證,怎麼說他。」寶玉道:「也罷,這件事我也應起來,就說是我唬他們頑的,悄悄的偷太太了的來了。兩件事都完了。」襲人道:「也倒是件陰騭事,保全人的賊名兒。只是太太聽見你又說你小孩子氣,不知好歹了。」平兒笑道:「這也倒是小事。如今便從趙姨娘屋裡起了贓來也容易,我只怕又傷著一個好人的體面。別人都別管,這一個人豈不又生氣。我可憐的是他,不肯為了打老鼠傷了玉瓶。」說著,把三個指頭一伸。襲人等聽說,便知他說的是探春。大家都忙說:「可是這話。竟是我們這裡應了起來的為是。」平兒又笑道:「也須得把彩雲和玉釧兒兩個業障叫了來,問準了他方好。不然他們得了益,不說為這個,倒象我沒了本事問不出來,煩出這裡來完事,他們以後越發偷的偷,不管的不管了。」襲人等笑道:「正是,也要你留下地步。

平兒便命人叫了他兩個來,說道:「不用慌,賊已有了。」玉釧兒先問賊在那裡,平兒道:「現在二奶奶屋裡,你問他什麼應什麼。我心裡明知不是他偷的,可憐他害怕都承認。這裡寶二爺不過意,要替他認一半。我待要說出來,但只是這做賊的素日又是和我好的一個姊妹,窩主卻是平常,裡面又傷著一個好人的體面,因此為難,少不得央求寶二爺應了,大家無事。如今反要問你們兩個,還是怎樣?若從此以後大家小心存體面,這便求寶二爺應了;若不然,我就回了二奶奶,別冤屈了好人。」彩雲聽了,不覺紅了臉,一時羞惡之心感發,便說道:「姐姐放心,也別冤了好人,也別帶累了無辜之人傷體面。偷東西原是趙姨奶奶央告我再三,我拿了些與環哥是情真。連太太在家我們還拿過,各人去送人,也是常事。我原說嚷過兩天就罷了。如今既冤屈了好人,我心也不忍。姐姐竟帶了我回奶奶去,我一概應了完事。」眾人聽了這話,一個個都詫異,他竟這樣有肝膽。寶玉忙笑道:「彩雲姐姐果然是個正經人。如今也不用你應,我只說是我悄悄的偷的唬你們頑,如今鬧出事來,我原該承認。只求姐姐們以後省些事,大家就好了。」彩雲道:「我乾的事為什麼叫你應,死活我該去受。」平兒襲人忙道:「不是這樣說,你一應了,未免又叨登出趙姨奶奶來,那時三姑娘聽了,豈不生氣。竟不如寶二爺應了,大家無事,且除這幾個人皆不得知道這事,何等的乾凈。但只以後千萬大家小心些就是了。要拿什麼,好歹奈到太太到家,那怕連這房子給了人,我們就沒干係了。」彩雲聽了,低頭想了一想,方依允。

開是大家商議妥貼,平兒帶了他兩個並芳官往前邊來,至上夜房中叫了五錢,將茯苓霜一節也悄悄的教他說系芳官所贈,五兒感謝不盡。平兒帶他們來至自己這邊,已見林之孝家的帶領了幾個媳婦,押解著柳家的等夠多時。林之孝家的又向平兒說:「今兒一早押了他來,恐園裡沒人伺候姑娘們的飯,我暫且將秦顯的女人派了去伺候。姑娘一併回明奶奶,他倒乾凈謹慎,以後就派他常伺候罷。」平兒道:「秦顯的女人是誰?我不大相熟。」林之孝家的道:「他是園裡南角子上夜的白日里沒什麼事,所以姑娘不大相識。高高孤拐,大大的眼睛,最乾凈爽利的。」玉釧兒道:「是了。姐姐,你怎麼忘了?他是跟二姑娘的司棋的嬸娘。司棋的父母雖是大老爺那邊的人,他這叔叔卻是咱們這邊的。」平兒聽了,方想起來,笑道:「哦,你早說是他,我就明白了。」又笑道:「也太派急了些。如今這事八下里水落石出了,連前兒太太屋裡丟的也有了主兒。是寶玉那日過來和這兩個業障要什麼的,偏這兩個業障慪他頑,說太太不在家不敢拿。寶玉便瞅他兩個不提防的時節,自己進去拿了些什麼出來。這兩個業障不知道,不唬慌了。如今寶玉聽見帶累了別人,方細細的告訴了我,拿出東西來我瞧,一件不差。那茯苓霜是寶玉外頭得了的,也曾賞過許多人,不話說平兒出來吩咐林之孝家的道:「大事化為小事,小事化為沒事,方是興旺之家。若得不了一點子小事,便揚鈴打鼓的亂折騰起來,不成道理。如今將他母女帶回,照舊去當差。將秦顯家的仍舊退回。再不必提此事。只是每日小心巡察要緊。」說畢,起身走了。柳家的母女忙向上磕頭,林家的帶回園中,回了李紈探春,二人皆說:「知道了,能可無事,很好。」

司棋等人空興頭了一陣。那秦顯家的好容易等了這個空子鑽了來,只興頭上半天。在廚房內正亂著接收家伙米糧煤炭等物,又查出許多虧空來,說:「粳米短了兩石,常用米又多支了一個月的,炭也欠著額數。」一面又打點送林之孝家的禮,悄悄的備了一簍炭,五百斤木柴,一擔粳米,在外邊就遣了子侄送入林家去了;又打點送帳房的禮;又預備幾樣菜蔬請幾位同事的人,說:「我來了,全仗列位扶持。自今以後都是一家人了。我有照顧不到的,好歹大家照顧些。」正亂著,忽有人來說與他:「看過這早飯就出去罷。柳嫂兒原無事,如今還交與他管了。」秦顯家的聽了。轟去魂魄,垂頭喪氣,登時掩旗息鼓,捲包而出。送人之物白丟了許多,自己倒要折變了賠補虧空。連司棋都氣了個倒仰,無計輓回,只得罷了。

趙姨娘正因彩雲私贈了許多東西,被玉釧兒吵出,生恐查詰出來,每日捏一把汗打聽信兒。忽見彩雲來告訴說:「都是寶玉應了,從此無事。」趙姨娘方把心放下來。誰知賈環聽如此說,便起了疑心,將彩雲凡私贈之物都拿了出來,照著彩雲的臉摔了去,說:「這兩面三刀的東西!我不稀罕。你不和寶玉好,他如何肯替你應。你既有擔當給了我,原該不與一個人知道。如今你既然告訴他,如今我再要這個,也沒趣兒。」彩雲見如此,急的發身賭誓,至於哭了,百般解說,賈環執意不信,說:「不看你素日之情,去告訴二嫂子,就說你偷來給我,我不敢要。你細想去。」說畢,摔手出去了。急的趙姨娘罵:「沒造化的種子,蛆心孽障。」氣的彩雲哭個淚乾腸斷。趙姨娘百般的安慰他:「好孩子,他辜負了你的心,我看的真。讓我收起來,過兩日他自然迴轉過來了。」說著,便要收東西。彩雲賭氣一頓包起來,乘人不見時,來至園中,都撇在河內,順水沉的沉漂的漂了。自己氣的夜間在被內暗哭。

當下又值寶玉生日已到,原來寶琴也是這日,二人相同。因王夫人不在家,也不曾象往年鬧熱。只有張道士送了四樣禮,換的寄名符兒;還有幾處僧尼廟的和尚姑子送了供尖兒,並壽星紙馬疏頭,並本命星官值年太歲周年換的鎖兒。家中常走的女先兒來上壽。王子騰那邊,仍是一套衣服,一雙鞋襪,一百壽桃,一百束上用銀絲掛麵。薛姨娘處減一等。其餘家中人,尤氏仍是一雙鞋襪;鳳姐兒是一個宮制四面和合荷包,裡面裝一個金壽星,一件波斯國所制玩器。各廟中遣人去放堂舍錢。又另有寶琴之禮,不能備述。姐妹中皆隨便,或有一扇的,或有一字的,或有一畫的,或有一詩的,聊復應景而已。

這日寶玉清晨起來,梳洗已畢,冠帶出來。至前廳院中,已有李貴等四五個人在那裡設下天地香燭,寶玉炷了香。行畢禮,奠茶焚紙後,便至寧府中宗祠祖先堂兩處行畢禮,出至月臺上,又朝上遙拜過賈母、賈政、王夫人等。一順到尤氏上房,行過禮,坐了一回,方回榮府。先至薛姨媽處,薛姨媽再三拉著,然後又遇見薛蝌,讓一回,方進園來。晴雯麝月二人跟隨,小丫頭夾著氈子,從李氏起,一一挨著,長的房中到過。復出二門,至李、趙、張、王四個奶媽家讓了一回,方進來。雖眾人要行禮,也不曾受。回至房中,襲人等只都來說一聲就是了。王夫人有言,不令年輕人受禮,恐折了福壽,故皆不磕頭。

歇一時,賈環賈蘭等來了,襲人連忙拉住,坐了一坐,便去了。寶玉笑說走乏了,便歪在床上。方吃了半盞茶,只聽外面咭咭呱呱,一群丫頭笑進來,原來是翠墨、小螺、翠縷、入畫,邢岫煙的丫頭篆兒,並奶子抱巧姐兒,彩鸞、繡鸞八九個人,都抱著紅氈笑著走來,說:「拜壽的擠破了門了,快拿面來我們吃。」剛進來時,探春、湘雲、寶琴、岫煙、惜春也都來了。寶玉忙迎出來,笑說:「不敢起動,快預備好茶。」進入房中,不免推讓一回,大家歸坐。襲人等捧過茶來,才吃了一口,平兒也打扮的花枝招展的來了。寶玉忙迎出來,笑說:「我方纔到鳳姐姐門上,回了進去,不能見,我又打發人進去讓姐姐的。」平兒笑道:「我正打發你姐姐梳頭,不得出來回你。後來聽見又說讓我,我那裡禁當的起,所以特趕來磕頭。」寶玉笑道:「我也禁當不起。」襲人早在外間安了坐,讓他坐。平兒便福下去,寶玉作揖不迭。平兒便跪下去,寶玉也忙還跪下,襲人連忙攙起來。又下了福,寶玉又還了一揖。襲人笑推寶玉:「你再作揖。」寶玉道:「已經完了,怎麼又作揖?」襲人笑道:「這是他來給你拜壽。今兒也是他的生日,你也該給他拜壽。」寶玉聽了,喜的忙作下揖去,說:「原來今兒也是姐姐的芳誕。」平兒還萬福不迭。湘雲拉寶琴岫煙說:「你們四個人對拜壽,直拜一天才是。」探春忙問:「原來邢妹妹也是今兒?我怎麼就忘了。」忙命丫頭:「去告訴二奶奶,趕著補了一分禮,與琴姑娘的一樣,送到二姑娘屋裡去。」丫頭答應著去了。岫煙見湘雲直口說出來,少不得要到各房去讓讓。

探春笑道:「倒有些意思,一年十二個月,月月有幾個生日。人多了,便這等巧,也有三個一日、兩個一日的。大年初一日也不白過,大姐姐占了去。怨不得他福大,生日比別人就占先。又是太祖太爺的生日。過了燈節,就是老太太和寶姐姐,他們娘兒兩個遇的巧。三月初一日是太太,初九日是璉二哥哥。二月沒人。」襲人道:「二月十二是林姑娘,怎麼沒人?就只不是咱家的人。」探春笑道:「我這個記性是怎麼了!」寶玉笑指襲人道:「他和林妹妹是一日,所以他記的。」探春笑道:「原來你兩個倒是一日。每年連頭也不給我們磕一個。平兒的生日我們也不知道,這也是才知道。」平兒笑道:「我們是那牌兒名上的人,生日也沒拜壽的福,又沒受禮職份,可吵鬧什麼,可不悄悄的過去。今兒他又偏吵出來了,等姑娘們回房,我再行禮去罷。」探春笑道:「也不敢驚動。只是今兒倒要替你過個生日,我心才過得去。」寶玉湘雲等一齊都說:「很是。」探春便吩咐了丫頭:「去告訴他奶奶,就說我們大家說了,今兒一日不放平兒出去,我們也大家湊了分子過生日呢。」丫頭笑著去了,半日,回來說:「二奶奶說了,多謝姑娘們給他臉。不知過生日給他些什麼吃,只別忘了二奶奶,就不來絮聒他了。」眾人都笑了。

探春因說道:「可巧今兒裡頭廚房不預備飯,一應下麵弄菜都是外頭收拾。咱們就湊了錢叫柳家的來攬了去,只在咱們裡頭收拾倒好。」眾人都說是極。探春一面遣人去問李紈、寶釵、黛玉,一面遣人去傳柳家的進來,吩咐他內廚房中快收拾兩桌酒席。柳家的不知何意,因說外廚房都預備了。探春笑道:「你原來不知道,今兒是平姑娘的華誕。外頭預備的是上頭的,這如今我們私下又湊了分子,單為平姑娘預備兩桌請他。你只管揀新巧的菜蔬預備了來,開了帳和我那裡領錢。」柳家的笑道:「原來今日也是平姑娘的千秋,我竟不知道。」說著,便向平兒磕下頭去,慌的平兒拉起他來。柳家的忙去預備酒席。

這裡探春又邀了寶玉,同到廳上去吃面,等到李紈寶釵一齊來全,又遣人去請薛姨媽與黛玉。因天氣和暖,黛玉之疾漸愈,故也來了。花團錦簇,擠了一廳的人。

誰知薛蝌又送了巾扇香帛四色壽禮與寶玉,寶玉於是過去陪他吃面。兩家皆治了壽酒,互相酬送,彼此同領。至午間,寶玉又陪薛蝌吃了兩杯酒。寶釵帶了寶琴過來與薛蝌行禮,把盞畢,寶釵因囑薛蝌:「家裡的酒也不用送過那邊去,這虛套竟可收了。你只請伙計們吃罷。我們和寶兄弟進去還要待人去呢,也不能陪你了。」 薛蝌忙說:「姐姐兄弟只管請,只怕伙計們也就好來了。」寶玉忙又告過罪,方同他姊妹回來。

一進角門,寶釵便命婆子將門鎖上,把鑰匙要了自己拿著。寶玉忙說:「這一道門何必關,又沒多的人走。況且姨娘、姐姐、妹妹都在裡頭,倘或家去取什麼,豈不費事。」寶釵笑道:「小心沒過逾的。你瞧你們那邊,這幾日七事八事,竟沒有我們這邊的人,可知是這門關的有功效了。若是開著,保不住那起人圖順腳,抄近路從這裡走,攔誰的是?不如鎖了,連媽和我也禁著些,大家別走。縱有了事,就賴不著這邊的人了。」寶玉笑道:「原來姐姐也知道我們那邊近日丟了東西?」 寶釵笑道:「你只知道玫瑰露和茯苓霜兩件,乃因人而及物。若非因人,你連這兩件還不知道呢。殊不知還有幾件比這兩件大的呢。若以後叨登不出來,是大家的造化;若叨登出來,不知裡頭連累多少人呢。你也是不管事的人,我才告訴你。平兒是個明白人,我前兒也告訴了他,皆因他奶奶不在外頭,所以使他明白了。若不出來,大家樂得丟開手。若犯出來,他心裡已有稿子,自有頭緒,就冤屈不著平人了。你只聽我說,以後留神小心就是了,這話也不可對第二個人講。」

說著,來到沁芳亭邊,只見襲人、香菱、待書、素雲、晴雯、麝月、芳官、蕊官、藕官等十來個人都在那裡看魚作耍。見他們來了,都說:「 藥欄里預備下了,快去上席罷。」寶釵等隨攜了他們同到了芍藥欄中紅香圃三間小敞廳內。連尤氏已請過來了,諸人都在那裡,只沒平兒。

原來平兒出去,有賴林諸家送了禮來,連三接四,上中下三等家人來拜壽送禮的不少,平兒忙著打發賞錢道謝,一面又色色的回明鳳姐兒,不過留下幾樣,也有不收的,也有收下即刻賞與人的。忙了一回,又直待鳳姐兒吃過面,方換了衣裳往園裡來。

剛進了園,就有幾個丫鬟來找他,一同到了紅香圃中。只見筵開玳瑁,褥設芙蓉。眾人都笑:「壽星全了。」上面四座定要讓他四個人坐,四人皆不肯。薛姨媽說:「我老天拔地,又不合你們的群兒,我倒覺拘的慌,不如我到廳上隨便躺躺去倒好。我又吃不下什麼去,又不大吃酒,這裡讓他們倒便宜。」尤氏等執意不從。寶釵道:「這也罷了,倒是讓媽在廳上歪著自如些,有愛吃的送些過去,倒自在了。且前頭沒人在那裡,又可照看了。」探春等笑道:「既這樣,恭敬不如從命。」 因大家送了他到議事廳上,眼看著命丫頭們鋪了一個錦褥並靠背引枕之類,又囑咐:「好生給姨媽捶腿,要茶要水別推三扯四的。回來送了東西來,姨媽吃了就賞你們吃。只別離了這裡出去。」小丫頭們都答應了。

探春等方回來。終久讓寶琴岫煙二人在上,平兒面西坐,寶玉面東坐。探春又接了鴛鴦來,二人並肩對面相陪。西邊一桌,寶釵黛玉湘雲迎春惜春,一面又拉了香菱玉釧兒二人打橫。三桌上,尤氏李紈又拉了襲人彩雲陪坐。四桌上便是紫鵑、鶯兒、晴雯、小螺、司棋等人圍坐。當下探春等還要把盞,寶琴等四人都說:「這一鬧,一日都坐不成了。」方纔罷了。兩個女先兒要彈詞上壽,眾人都說:「我們沒人要聽那些野話,你廳上去說給姨太太解悶兒去罷。」一面又將各色吃食揀了,命人送與薛姨媽去。

寶玉便說:「雅坐無趣,須要行令才好。」眾人有的說行這個令好,那個又說行那個令好。黛玉道:「依我說,拿了筆硯將各色全都寫了,拈成鬮兒,咱們抓出那個來,就是那個。」眾人都道妙。即拿了一副筆硯花箋。香菱近日學了詩,又天天學寫字,見了筆硯便圖不得,連忙起座說:「我寫。」大家想了一回,共得了十來個,念著,香菱一一的寫了,搓成鬮兒,擲在一個瓶中間。探春便命平兒揀,平兒向內攪了一攪,用箸拈了一個出來,打開看,上寫著「射覆」二字。寶釵笑道: 「把個酒令的祖宗拈出來。『射覆』從古有的,如今失了傳,這是後人纂的,比一切的令都難。這裡頭倒有一半是不會的,不如毀了,另拈一個雅俗共賞的。」探春笑道:「既拈了出來,如何又毀。如今再拈一個,若是雅俗共賞的,便叫他們行去。咱們行這個。」說著又著襲人拈了一個,卻是「拇戰」。史湘雲笑著說:「這個簡斷爽利,合了我的脾氣。我不行這個『射覆』,沒的垂頭喪氣悶人,我只划拳去了。」探春道:「惟有他亂令,寶姐姐快罰他一鐘。」寶釵不容分說,便灌湘雲一杯。

探春道:「我吃一杯,我是令官,也不用宣,只聽我分派。」命取了令骰令盆來,「從琴妹擲起,挨下擲去,對了點的二人射覆。」寶琴一擲,是個三,岫煙寶玉等皆擲的不對,直到香菱方擲了一個三。寶琴笑道:「只好室內生春,若說到外頭去,可太沒頭緒了。」探春道:「自然。三次不中者罰一杯。你覆,他射。」寶琴想了一想,說了個「老」字。香菱原生於這令,一時想不到,滿室滿席都不見有與「老」字相連的成語。湘雲先聽了,便也亂看,忽見門斗上貼著「紅香圃」三個字,便知寶琴覆的是「吾不如老圃」的「圃」字。見香菱射不著,眾人擊鼓又催,便悄悄的拉香菱,教他說「藥」字。黛玉偏看見了,說「快罰他,又在那裡私相傳遞呢。」哄的眾人都知道了,忙又罰了一杯,恨的湘雲拿筷子敲黛玉的手。於是罰了香菱一杯。下則寶釵和探春對了點子。探春便覆了一個「人」字。寶釵笑道:「這個『人』字泛的很。」探春笑道:「添一字,兩覆一射也不泛了。」說著,便又說了一個「窗」字。寶釵一想,因見席上有雞,便射著他是用「雞窗」 「雞人」二典了,因射了一個「塒」字。探春知他射著,用了「雞棲於塒」的典,二人一笑,各飲一口門杯。

湘雲等不得,早和寶玉「三」「五」亂叫,划起拳來。那邊尤氏和鴛鴦隔著席也「七」「八」亂叫划起來。平兒襲人也作了一對划拳,叮叮噹當只聽得腕上的鐲子響。一時湘雲贏了寶玉,襲人贏了平兒,尤氏贏了鴛鴦,三個人限酒底酒面,湘雲便說:「酒面要一句古文,一句舊詩,一句骨牌名,一句曲牌名,還要一句時憲書上的話,共總湊成一句話。酒底要關人事的果菜名。」眾人聽了,都笑說:「惟有他的令也比人嘮叨,倒也有意思。」便催寶玉快說。寶玉笑道:「誰說過這個,也等想一想兒。」黛玉便道:「你多喝一鐘,我替你說。」寶玉真個喝了酒,聽黛玉說道:

\begin{center}

落霞與孤鶩齊飛,風急江天過雁哀,卻是一隻折足雁,叫的人九迴腸,這是鴻雁來賓。

\end{center}

說的大家笑了,說:「這一串子倒有些意思。」黛玉又拈了一個榛穰,說酒底道:

\begin{center}

榛子非關隔院砧,何來萬戶搗衣聲。

\end{center}

令完,鴛鴦襲人等皆說的是一句俗語,都帶一個「壽」字的,不能多贅。

大家輪流亂划了一陣,這上面湘雲又和寶琴對了手,李紈和岫煙對了點子。李紈便覆了一個「瓢」字,岫煙便射了一個「綠」字,二人會意,各飲一口。湘雲的拳卻輸了,請酒面酒底。寶琴笑道:「請君入瓮。」大家笑起來,說:「這個典用的當。」湘雲便說道:

\begin{center}

奔騰而砰湃,江間波浪兼天涌,須要鐵鎖纜孤舟,既遇著一江風,不宜出行。

\end{center}

說的眾人都笑了,說:「好個謅斷了腸子的。怪道他出這個令,故意惹人笑。」又聽他說酒底。湘雲吃了酒,揀了一塊鴨肉呷口,忽見碗內有半個鴨頭,遂揀了出來吃腦子。眾人催他:「別隻顧吃,到底快說了。」湘雲便用箸子舉著說道:

\begin{center}

這鴨頭不是那丫頭,頭上那討桂花油。

\end{center}

眾人越發笑起來,引的晴雯、小螺、鶯兒等一干人都走過來說:「雲姑娘會開心兒,拿著我們取笑兒,快罰一杯才罷。怎見得我們就該擦桂花油的?倒得每人給一瓶子桂花油擦擦。」黛玉笑道:「他倒有心給你們一瓶子油,又怕掛誤著打盜竊的官司。」眾人不理論,寶玉卻明白,忙低了頭。彩雲有心病,不覺的紅了臉。寶釵忙暗暗的瞅了黛玉一眼。黛玉自悔失言,原是趣寶玉的,就忘了趣著彩雲。自悔不及,忙一頓行令划拳岔開了。

底下寶玉可巧和寶釵對了點子。寶釵覆了一個「寶」字,寶玉想了一想,便知是寶釵作戲指自己所佩通靈玉而言,便笑道:「姐姐拿我作雅謔,我卻射著了。說出來姐姐別惱,就是姐姐的諱『釵』字就是了。」眾人道:「怎麼解?」寶玉道:「他說『寶』,底下自然是『玉』了。我射『釵』字,舊詩曾有『敲斷玉釵紅燭冷』,豈不射著了。」湘雲說道:「這用時事卻使不得,兩個人都該罰。」香菱忙道:「不止時事,這也有出處。」湘雲道:「『寶玉』二字並無出處,不過是春聯上或有之,詩書紀載並無,算不得。」香菱道:「前日我讀岑嘉州五言律,現有一句說『此鄉多寶玉』,怎麼你倒忘了?後來又讀李義山七言絕句,又有一句『寶釵無日不生塵』,我還笑說他兩個名字都原來在唐詩上呢。」眾人笑說:「這可問住了,快罰一杯。」湘雲無語,只得飲了。大家又該對點的對點,划拳的划拳。這些人因賈母王夫人不在家,沒了管束,便任意取樂,呼三喝四,喊七叫八。滿廳中紅飛翠舞,玉動珠搖,真是十分熱鬧。頑了一回,大家方起席散了一散,倏然不見了湘雲,只當他外頭自便就來,誰知越等越沒了影響,使人各處去找,那裡找得著。

接著林之孝家的同著幾個老婆子來,生恐有正事呼喚,二者恐丫鬟們年青,乘王夫人不在家不服探春等約束,姿意痛飲,失了體統,故來請問有事無事。探春見他們來了,便知其意,忙笑道:「你們又不放心,來查我們來了。我們沒有多吃酒,不過是大家頑笑,將酒作個引子,媽媽們別耽心。」李紈尤氏都也笑說:「你們歇著去罷,我們也不敢叫他們多吃了。」林之孝家的等人笑說:「我們知道,連老太太叫姑娘吃酒姑娘們還不肯吃,何況太太們不在家,自然頑罷了。我們怕有事,來打聽打聽。二則天長了,姑娘們頑一回子還該點補些小食兒。素日又不大吃雜東西,如今吃一兩杯酒,若不多吃些東西,怕受傷。」探春笑道:「媽媽們說的是,我們也正要吃呢。」因回頭命取點心來。兩旁丫鬟們答應了,忙去傳點心。探春又笑讓:「你們歇著去罷,或是姨媽那裡說話兒去。我們即刻打發人送酒你們吃去。」林之孝家的等人笑回:「不敢領了。」又站了一回,方退了出來。平兒摸著臉笑道:「我的臉都熱了,也不好意思見他們。依我說竟收了罷,別惹他們再來,倒沒意思了。」探春笑道:「不相干,橫豎咱們不認真喝酒就罷了。」

正說著,只見一個小丫頭笑嘻嘻的走來:「姑娘們快瞧雲姑娘去,吃醉了圖涼快,在山子後頭一塊青板石凳上睡著了。」眾人聽說,都笑道:「快別吵嚷。」說著,都走來看時,果見湘雲卧於山石僻處一個石凳子上,業經香夢沉酣,四面芍藥花飛了一身,滿頭臉衣襟上皆是紅香散亂,手中的扇子在地下,也半被落花埋了,一群蜂蝶鬧穰穰的圍著他,又用鮫帕包了一包芍藥花瓣枕著。眾人看了,又是愛,又是笑,忙上來推喚輓扶。湘雲口內猶作睡語說酒令,唧唧嘟嘟說:

\begin{center}

泉香而酒冽,玉盞盛來琥珀光,直飲到梅梢月上,醉扶歸,卻為宜會親友。

\end{center}

眾人笑推他,說道:「快醒醒兒吃飯去,這潮凳上還睡出病來呢。」湘雲慢啟秋波,見了眾人,低頭看了一看自己,方知是醉了。原是來納涼避靜的,不覺的因多罰了兩杯酒,嬌嫋不勝,便睡著了,心中反覺自愧。連忙起身扎掙著同人來至紅香圃中,用過水,又吃了兩盞釅茶。探春忙命將醒酒石拿來給他銜在口內,一時又命他喝了一些酸湯,方纔覺得好了些。

當下又選了幾樣果菜與鳳姐送去,鳳姐兒也送了幾樣來。寶釵等吃過點心,大家也有坐的,也有立的,也有在外觀花的,也有扶欄觀魚的,各自取便說笑不一。探春便和寶琴下棋,寶釵岫煙觀局。林黛玉和寶玉在一簇花下唧唧噥噥不知說些什麼。只見林之孝家的和一群女人帶了一個媳婦進來。那媳婦愁眉苦臉,也不敢進廳,只到了階下,便朝上跪下了,碰頭有聲。探春因一塊棋受了敵,算來算去總得了兩個眼,便折了官著,兩眼只瞅著棋枰,一隻手卻伸在盒內,只管抓弄棋子作想,林之孝家的站了半天,因回頭要茶時才看見,問:「什麼事?」林之孝家的便指那媳婦說:「這是四姑娘屋裡的小丫頭彩兒的娘,現是園內伺候的人。嘴很不好,才是我聽見了問著他,他說的話也不敢回姑娘,竟要攆出去才是。」探春道:「怎麼不回大奶奶?」林之孝家的道:「方纔大奶奶都往廳上姨太太處去了,頂頭看見,我已回明白了,叫回姑娘來。」探春道:「怎麼不回二奶奶?」平兒道:「不回去也罷,我回去說一聲就是了。」探春點點頭,道:「既這麼著,就攆出他去,等太太來了,再回定奪。」說畢仍又下棋。這林之孝家的帶了那人去不提。

黛玉和寶玉二人站在花下,遙遙知意。黛玉便說道:「你家三丫頭倒是個乖人。雖然叫他管些事,倒也一步兒不肯多走。差不多的人就早作起威福來了。」寶玉道:「你不知道呢。你病著時,他幹了好幾件事。這園子也分了人管,如今多掐一草也不能了。又蠲了幾件事,單拿我和鳳姐姐作筏子禁別人。最是心裡有算計的人,豈只乖而已。」黛玉道:「要這樣才好,咱們家裡也太花費了。我雖不管事,心裡每常閑了,替你們一算計,出的多進的少,如今若不省儉,必致後手不接。」 寶玉笑道:「憑他怎麼後手不接,也短不了咱們兩個人的。」黛玉聽了,轉身就往廳上尋寶釵說笑去了。

寶玉正欲走時,只見襲人走來,手內捧著一個小連環洋漆茶盤,裡面可式放著兩鐘新茶,因問:「他往那去了?我見你兩個半日沒吃茶,巴巴的倒了兩鐘來,他又走了。」寶玉道:「那不是他,你給他送去。」說著自拿了一鐘。襲人便送了那鐘去,偏和寶釵在一處,只得一鐘茶,便說:「那位渴了那位先接了,我再倒去。」寶釵笑道:「我卻不渴,只要一口漱一漱就夠了。」說著先拿起來喝了一口,剩下半杯遞在黛玉手內。襲人笑說:「我再倒去。」黛玉笑道:「你知道我這病,大夫不許我多吃茶,這半鐘盡夠了,難為你想的到。」說畢,飲乾,將杯放下。襲人又來接寶玉的。寶玉因問:「這半日沒見芳官,他在那裡呢?」襲人四顧一瞧說:「才在這裡幾個人鬥草的,這會子不見了。」

寶玉聽說,便忙回至房中,果見芳官面向里睡在床上。寶玉推他說道:「快別睡覺,咱們外頭頑去,一回兒好吃飯的。」芳官道:「你們吃酒不理我,教我悶了半日,可不來睡覺罷了。」寶玉拉了他起來,笑道:「咱們晚上家裡再吃,回來我叫襲人姐姐帶了你桌上吃飯,何如?」芳官道:「藕官蕊官都不上去,單我在那裡也不好。我也不慣吃那個麵條子,早起也沒好生吃。才剛餓了,我已告訴了柳嫂子,先給我做一碗湯盛半碗粳米飯送來,我這裡吃了就完事。若是晚上吃酒,不許教人管著我,我要儘力吃夠了才罷。我先在家裡,吃二三斤好惠泉酒呢。如今學了這勞什子,他們說怕壞嗓子,這幾年也沒聞見。乘今兒我是要開齋了。」寶玉道: 「這個容易。」

說著,只見柳家的果遣了人送了一個盒子來。小燕接著揭開,裡面是一碗蝦丸雞皮湯,又是一碗酒釀清蒸鴨子,一碟腌的胭脂鵝脯,還有一碟四個奶油松瓤捲酥,並一大碗熱騰騰碧熒熒蒸的綠畦香稻粳米飯。小燕放在案上,走去拿了小菜並碗箸過來,撥了一碗飯。芳官便說:「油膩膩的,誰吃這些東西。」只將湯泡飯吃了一碗,揀了兩塊腌鵝就不吃了。寶玉聞著,倒覺比往常之味有勝些似的,遂吃了一個捲酥,又命小燕也撥了半碗飯,泡湯一吃,十分香甜可口。小燕和芳官都笑了。吃畢,小燕便將剩的要交回。寶玉道:「你吃了罷,若不夠再要些來。」小燕道:「不用要,這就夠了。方纔麝月姐姐拿了兩盤子點心給我們吃了,我再吃了這個,盡不用再吃了。」說著,便站在桌旁一頓吃了,又留下兩個捲酥,說:「這個留著給我媽吃。晚上要吃酒,給我兩碗酒吃就是了。」寶玉笑道:「你也愛吃酒?等著咱們晚上痛喝一陣。你襲人姐姐和晴雯姐姐量也好,也要喝,只是每日不好意思。今兒大家開齋。還有一件事,想著囑咐你,我竟忘了,此刻才想起來。以後芳官全要你照看他,他或有不到的去處,你提他,襲人照顧不過這些人來。」小燕道:「我都知道,都不用操心。但只這五兒怎麼樣?」寶玉道:「你和柳家的說去,明兒直叫他進來罷,等我告訴他們一聲就完了。」芳官聽了,笑道:「這倒是正經。」小燕又叫兩個小丫頭進來,伏侍洗手倒茶,自己收了家伙,交與婆子,也洗了手,便去找柳家的,不在話下。

寶玉便出來,仍往紅香圃尋眾姐妹,芳官在後拿著巾扇。剛出了院門,只見襲人晴雯二人攜手回來。寶玉問:「你們做什麼?」襲人道:「擺下飯了,等你吃飯呢。」寶玉便笑著將方纔吃的飯一節告訴了他兩個。襲人笑道:「我說你是貓兒食,聞見了香就好,隔鍋飯兒香。雖然如此,也該上去陪他們多少應個景兒。」晴雯用手指戳在芳官額上,說道:「你就是個狐媚子,什麼空兒跑了去吃飯,兩個人怎麼就約下了,也不告訴我們一聲兒。」襲人笑道:「不過是誤打誤撞的遇見了,說約下了可是沒有的事。」晴雯道:「既這麼著,要我們無用。明兒我們都走了,讓芳官一個人就夠使了。」襲人笑道:「我們都去了使得,你卻去不得。」晴雯道: 「惟有我是第一個要去,又懶又笨,性子又不好,又沒用。」襲人笑道:「倘或那孔雀褂子再燒個窟窿,你去了誰可會補呢。你倒別和我拿三撇四的,我煩你做個什麼,把你懶的橫針不拈,豎線不動。一般也不是我的私活煩你,橫豎都是他的,你就都不肯做。怎麼我去了幾天,你病的七死八活,一夜連命也不顧給他做了出來,這又是什麼原故?你到底說話,別隻佯憨,和我笑,也當不了什麼。」大家說著,來至廳上。薛姨媽也來了。大家依序坐下吃飯。寶玉只用茶泡半碗飯,應景而已。一時吃畢,大家吃茶閑話,又隨便頑笑。

外面小螺和香菱、芳官、蕊官、藕官、荳官等四五個人,都滿園中頑了一回,大家採了些花草來兜著,坐在花草堆中鬥草。這一個說:「我有觀音柳。」那一個說:「我有羅漢松。」那一個又說:「我有君子竹。」這一個又說:「我有美人蕉。」這個又說:「我有星星翠。」那個又說:「我有月月紅。」這個又說:「我有《牡丹亭》畔的牡丹叶。」那個又說:「我有《琵琶記》里的枇杷果。」荳官便說:「我有姐妹花。」眾人沒了,香菱便說:「我有夫妻蕙。」荳官說:「從沒聽見有個夫妻蕙。」香菱道:「一箭一花為蘭,一箭數花為蕙。凡蕙有兩枝,上下結花者為兄弟蕙,有並頭結花者為夫妻蕙。我這枝並頭的,怎麼不是。」荳官沒的說了,便起身笑道:「依你說,若是這兩枝一大一小,就是老子兒子蕙了。若兩枝背面開的,就是仇人蕙了。你漢子去了大半年,你想夫妻了?便扯上蕙也有夫妻,好不害羞!」香菱聽了,紅了臉,忙要起身擰他,笑罵道:「我把你這個爛了嘴的小蹄子!滿嘴裡汗□的胡說了。等我起來打不死你這小蹄子!」荳官見他要勾來,怎容他起來,便忙連身將他壓倒。回頭笑著央告蕊官等:「你們來,幫著我擰他這謅嘴。」兩個人滾在草地下。眾人拍手笑說:「了不得了,那是一窪子水,可惜污了他的新裙子了。」荳官回頭看了一看,果見旁邊有一汪積雨,香菱的半扇裙子都污濕了,自己不好意思,忙奪了手跑了。眾人笑個不住,怕香菱拿他們出氣,也都哄笑一散。

香菱起身低頭一瞧,那裙上猶滴滴點點流下綠水來。正恨罵不絕,可巧寶玉見他們鬥草,也尋了些花草來湊戲,忽見眾人跑了,只剩了香菱一個低頭弄裙,因問:「怎麼散了?」香菱便說:「我有一枝夫妻蕙,他們不知道,反說我謅,因此鬧起來,把我的新裙子也臟了。」寶玉笑道:「你有夫妻蕙,我這裡倒有一枝並蒂菱。」口內說,手內卻真個拈著一枝並蒂菱花,又拈了那枝夫妻蕙在手內。香菱道:「什麼夫妻不夫妻,並蒂不並蒂,你瞧瞧這裙子。」寶玉方低頭一瞧,便噯呀了一聲,說:「怎麼就拖在泥里了?可惜這石榴紅綾最不經染。」香菱道:「這是前兒琴姑娘帶了來的。姑娘做了一條,我做了一條,今兒才上身。」寶玉跌腳嘆道: 「若你們家,一日遭踏這一百件也不值什麼。只是頭一件既系琴姑娘帶來的,你和寶姐姐每人才一件,他的尚好,你的先臟了,豈不辜負他的心。二則姨媽老人家嘴碎,饒這麼樣,我還聽見常說你們不知過日子,只會遭踏東西,不知惜福呢。這叫姨媽看見了,又說一個不清。」香菱聽了這話,卻碰在心坎兒上,反倒喜歡起來了,因笑道:「就是這話了。我雖有幾條新裙子,都不和這一樣的,若有一樣的,趕著換了,也就好了。過後再說。」寶玉道:「你快休動,只站著方好,不然連小衣兒膝褲鞋面都要拖臟。我有個主意:襲人上月做了一條和這個一模一樣的,他因有孝,如今也不穿。竟送了你換下這個來,如何?」香菱笑著搖頭說:「不好。他們倘或聽見了倒不好。」寶玉道:「這怕什麼。等他們孝滿了,他愛什麼難道不許你送他別的不成。你若這樣,還是你素日為人了!況且不是瞞人的事,只這告訴寶姐姐也可,只不過怕姨媽老人家生氣罷了。」香菱想了一想有理,便點頭笑道:「就是這樣罷了,別辜負了你的心。我等著你,千萬叫他親自送來才好。」

寶玉聽了,喜歡非常,答應了忙忙的回來,一壁里低頭心下暗算:「可惜這麼一個人,沒父母,連自己本姓都忘了,被人拐出來,偏又賣與了這個霸王。」因又想起上日平兒也是意外想不到的,今日更是意外之意外的事了。一壁胡思亂想,來至房中,拉了襲人,細細告訴了他原故。香菱之為人,無人不憐愛的。襲人又本是個手中撒漫的,況與香菱素相交好,一聞此信,忙就開箱取了出來折好,隨了寶玉來尋著香菱,他還站在那裡等呢。襲人笑道:「我說你太淘氣了,足的淘出個故事來才罷。」香菱紅了臉,笑說:「多謝姐姐了,誰知那起促狹鬼使黑心。」說著,接了裙子,展開一看,果然同自己的一樣。又命寶玉背過臉去,自己叉手向內解下來,將這條繫上。襲人道:「把這臟了的交與我拿回去,收拾了再給你送來。你若拿回去,看見了也是要問的。」香菱道:「好姐姐,你拿去不拘給那個妹妹罷。我有了這個,不要他了。」襲人道:「你倒大方的好。」香菱忙又萬福道謝,襲人拿了臟裙便走。

香菱見寶玉蹲在地下,將方纔的夫妻蕙與並蒂菱用樹枝兒摳了一個坑,先抓些落花來鋪墊了,將這菱蕙安放好,又將些落花來掩了,方撮土掩埋平服。香菱拉他的手,笑道:「這又叫做什麼?怪道人人說你慣會鬼鬼祟祟使人肉麻的事。你瞧瞧,你這手弄的泥烏苔滑的,還不快洗去。」寶玉笑著,方起身走了去洗手,香菱也自走開。二人已走遠了數步,香菱復轉身回來叫住寶玉。寶玉不知有何話,扎著兩隻泥手,笑嘻嘻的轉來問:「什麼?」香菱只顧笑。因那邊他的小丫頭臻兒走來說:「二姑娘等你說話呢。」香菱方向寶玉道:「裙子的事可別向你哥哥說才好。」說畢,即轉身走了。寶玉笑道:「可不我瘋了,往虎口裡探頭兒去呢。」說著,也回去洗手去了。不知端詳,且聽下回分解。獨園內人有,連媽媽子們討了出去給親戚們吃,又轉送人,襲人了曾給過芳官之流的人。他們私情各相來往,也是常事。前兒那兩簍還擺在議事廳上,好好的原封沒動,怎麼就混賴起人來。等我回了奶奶再說。」說畢,抽身進了卧房,將此事照前言回了鳳姐兒一遍。

鳳姐兒道:「雖如此說,但寶玉為人不管青紅皂白愛兜攬事情。別人再求求他去,他又擱不住人兩句好話,給他個炭簍子戴上,什麼事他不應承。咱們若信了,將來若大事也如此,如何治人。還要細細的追求才是。依我的主意,把太太屋裡的丫頭都拿來,雖不便擅加拷打,只叫他們墊著磁瓦子跪在太陽地下,茶飯也別給吃。一日不說跪一日,便是鐵打的,一日也管招了。又道是『蒼蠅不抱無縫的蛋』。雖然這柳家的沒偷,到底有些影兒,人才說他。雖不加賊刑,也革出不用。朝廷家原有掛誤的,倒也不算委屈了他。」平兒道:「何苦來操這心!『得放手時須放手』,什麼大不了的事,樂得不施恩呢。依我說,縱在這屋裡操上一百分的心,終久咱們是那邊屋裡去的。沒的結些小人仇恨,使人含怨。況且自己又三災八難的,好容易懷了一個哥兒,到了六七個月還掉了,焉知不是素日操勞太過,氣惱傷著的。如今乘早兒見一半不見一半的,也倒罷了。」一席話,說的鳳姐兒倒笑了,說道:「憑你這小蹄子發放去罷。我才精爽些了,沒的淘氣。」平兒笑道:「這不是正經!」說畢,轉身出來,一一發放。要知端的,且聽下回分解。
